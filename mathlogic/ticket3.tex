\section{Ticket 3: ИИВ}
\label{sec-5}
\subsection{ИИВ, структура, модель}
\label{sec-5-1}

Сигнатура -- $(R, F, C, r)$: $R$ -- множество символов для предикатов, $F$ -- функциональных символов, $C$ -- символов констант, $r$ -- функция, определяющая арность $x \in R \cup F$. 

Интерпретация -- это приписывание символам значения и правил действия.

Структура -- это носитель $M$ (множество истинностных значений), сигнатура и интерпретация над носителем.

Если все аксиомы верны, то структура корректна.
В таком случае она называется моделью.

Выкидываем 10 аксиому, добавляем $\alpha \rightarrow \neg \alpha \rightarrow \beta$.

Она доказывается и в ИВ.
\begin{lemma} 
$\alpha, \alpha \vee \neg \alpha, \neg \alpha \vdash \beta$
\end{lemma}
\begin{tabular}{lll}
(1) & $\alpha$& Допущение\\
(2) & $\neg \alpha$& Допущение\\
(3) & $\alpha \rightarrow \neg \beta \rightarrow \alpha$& Сх. акс. 1\\
(4) & $\neg \beta \rightarrow \alpha$& M.P. 1,3\\
(5) & $\neg \alpha \rightarrow \neg \beta \rightarrow \neg \alpha$& Сх. акс. 1\\
(6) & $\neg \beta \rightarrow \neg \alpha$& M.P. 2,5\\
(7) & $(\neg \beta \rightarrow \alpha) \rightarrow (\neg \beta \rightarrow \neg \alpha) \rightarrow (\neg \neg \beta)$& Сх. акс. 9\\
(8) & $(\neg \beta \rightarrow \neg \alpha) \rightarrow (\neg \neg \beta)$& M.P. 4,7\\
(9) & $\neg \neg \beta$& M.P. 6,8\\
(10) & $\neg \neg \beta \rightarrow \beta$& Сх. акс. 10\\
(11) & $\beta$& M.P. 9,10\\
\end{tabular}


Таким образом мы умеем доказывать $\alpha \rightarrow \alpha \vee \neg \alpha \rightarrow \neg \alpha \rightarrow \beta$ применив 3 раза теорему о дедукции
\begin{lemma}
$\alpha \rightarrow \alpha \vee \neg \alpha \rightarrow \neg \alpha \rightarrow \beta, \alpha \vee \neg \alpha \vdash \alpha \rightarrow \neg \alpha \rightarrow \beta$
\end{lemma}
\begin{tabular}{lll}
(1) & $(\alpha \rightarrow \alpha \vee \neg \alpha) \rightarrow (\alpha \rightarrow \alpha \vee \neg \alpha \rightarrow (\neg \alpha \rightarrow \beta)) \rightarrow (\alpha \rightarrow (\neg \alpha \rightarrow \beta))$& Сх. акс. 2\\
(2) & $\alpha \vee \neg \alpha \rightarrow \alpha \rightarrow \alpha \vee \neg \alpha$& Сх. акс. 1\\
(3) & $\alpha \vee \neg \alpha$& Допущение\\
(4) & $\alpha \rightarrow \alpha \vee \neg \alpha$& M.P. 3,2\\
(5) & $(\alpha \rightarrow \alpha \vee \neg \alpha \rightarrow (\neg \alpha \rightarrow \beta)) \rightarrow (\alpha \rightarrow (\neg \alpha \rightarrow \beta))$& M.P. 4,1\\
(6) & $\alpha \rightarrow \alpha \vee \neg \alpha \rightarrow \neg \alpha \rightarrow \beta$& Допущение\\
(7) & $\alpha \rightarrow \neg \alpha \rightarrow \beta$& M.P. 6,5\\
\end{tabular}

\subsection{Опровергаемость исключенного третьего}
\label{sec-5-2}
Вводим в наше множество \emph{истинностных значений} дополнительный элемент \texttt{Н} (сокращение от слова <<Неизвестно>>). Отождествим \texttt{Н} с $\frac{1}{2}$, так что $\texttt{Л} < \texttt{Н} < \texttt{И}$. Определим операции на этом множестве \emph{истинностных значений}:
\begin{itemize}
\item конъюнкция: минимум из двух значений (например $\texttt{И} \& \texttt{Н} = \texttt{Н}$).
\item дизъюнкция: максимум из двух значений (например $\texttt{И} \vee \texttt{Н} = \texttt{И}$).
\item импликация: $\texttt{И} \rightarrow \alpha = \alpha$, $\texttt{Л} \rightarrow \alpha = \texttt{И}$, $\texttt{Н} \rightarrow \texttt{Л} = \texttt{Л}$, $\texttt{Н} \rightarrow \texttt{Н} = \texttt{И}$, $\texttt{Н} \rightarrow \texttt{И} = \texttt{И}$.
\item отрицание: $\neg \texttt{Н} = \texttt{Л}$, а для остальных элементов все так же.
\end{itemize}

Назовем формулу \emph{3-тавтологией}, если она принимает значение
\texttt{И} при любых значениях переменных из множества $\{\texttt{И}, \texttt{Л}\, \texttt{Н}\}$. Теперь нужно всего-лишь проверить, что все аксиомы являются 3-тавтологиями и, что если посылка импликации является тавтологией, то и заключение является тавтологией. Второе очевидно по определению тавтологии, а аксиомы просто проверяются вручную.

Значит любая интуиционистски выводимая формула 3-тавтология. Теперь заметим, что формула $\alpha \vee \neg \alpha$ принимает значение \texttt{Н} при $\alpha = \texttt{Н}$. Следовательно она не 3-тавтология, а значит невыводима.
\subsection{Решетки}
\label{sec-5-3}
Просто \emph{решетка} -- это $(L, +, *)$ в алгебраическом смысле и $(L, \leq)$ в порядковом. Решетку можно определить как алгебраическую структуру через аксиомы: 
\begin{itemize}
\item Аксиомы идемпотентность\\
$\alpha + \alpha = \alpha$\\
$\alpha * \alpha = \alpha$
\item Аксиомы коммутативности\\
$\alpha + \beta = \beta + \alpha$\\
$\alpha * \beta = \beta * \alpha$
\item Аксиомы ассоциативности\\
$(\alpha + \beta) + \gamma = \alpha + (\beta + \gamma)$\\
$(\alpha * \beta) * \gamma = \alpha * (\beta * \gamma)$
\item Аксиомы поглощения\\
$\alpha + (\alpha * \beta) = \alpha$\\
$\alpha * (\alpha + \beta) = \alpha$
\end{itemize} 
Также решетку можно определить как упорядоченное множество с частичным порядком на нем. Тогда операции $+, *$ определяются как $\sup$ и $\inf$
\begin{gather*}
\sup(\phi) = \min \lbrace u \mid u \geq \forall x \in \phi \rbrace\\
\inf(\phi) = \max \lbrace u \mid u \leq \forall x \in \phi \rbrace\\
\alpha + \beta = \sup (\lbrace \alpha, \beta \rbrace)\\
\alpha * \beta = \inf (\lbrace \alpha, \beta \rbrace)\\
\end{gather*}
Если для любых двух элементов из множества $S$ можно определить эти две операции, то $S$ называется решеткой.

\emph{Дистрибутивная решетка} -- \emph{решетка}, в которой добавляется дистрибутивность:
\[\alpha * (\beta + \gamma) = \alpha * \beta + \alpha * \gamma\]

\emph{Импликативная решетка} -- \emph{решетка}, в которой для любых двух элементов $\alpha$ и $\beta$ из множества существует псевдодополнение $\alpha$ относительно $\beta$ ($\alpha \rightarrow \beta$), которое определяется так:
\[\alpha \rightarrow \beta = max \lbrace \gamma \vert \gamma * \alpha \leq \beta \rbrace\]

Свойства \emph{импликативной решетки}:
\begin{itemize}
\item Существует максимальный элемент $\alpha \rightarrow \alpha$, обычно обозначаемый как $1$
\item Всякая \emph{импликативная решетка} дистрибутивна
\end{itemize}

\subsection{Алгебра Гейтинга, булева алгебра}
\label{sec-5-4}
\emph{Булева алгебра} -- $(L, +, *, -, 0, 1)$, с аксиомами:
\begin{itemize}
\item Аксиомы коммутативности\\
$\alpha + \beta = \beta + \alpha$\\
$\alpha * \beta = \beta * \alpha$
\item Аксиомы ассоциативности\\
$(\alpha + \beta) + \gamma = \alpha + (\beta + \gamma)$\\
$(\alpha * \beta) * \gamma = \alpha * (\beta * \gamma)$
\item Аксиомы поглощения\\
$\alpha + (\alpha * \beta) = \alpha$\\
$\alpha * (\alpha + \beta) = \alpha$
\item Аксиомы дистрибутивности\\
$\alpha + (\beta * \gamma) = (\alpha + \beta) * (\alpha + \gamma)$\\
$\alpha * (\beta + \gamma) = (\alpha * \beta) + (\alpha * \gamma)$
\item Аксиомы дополнительности\\
$\alpha * \neg \alpha = 0$\\
$\alpha + \neg \alpha = 1$
\end{itemize}

Также \emph{булеву алгебру} можно определить как импликативную решетку над фундированным множеством. Тогда $1$ в ней будет $\alpha \rightarrow \alpha$, $\neg \alpha = \alpha \rightarrow 0$. Тогда $\alpha * \neg \alpha = 0$ будет уже свойством, а $\alpha + \neg \alpha = 1$ все еще аксиомой.

\emph{Псевдобулева алгебра} (алгебра Гейтинга) -- это импликативная решетка над фундированным множеством с $\neg \alpha = \alpha \rightarrow 0$ (нет аксиомы $\alpha + \neg \alpha = 1$)
\subsection{Алгебра Линденбаума-Тарского}
\label{sec-5-5}
Пусть $V$ -- множество формул ИИВ\\
Порядок для решетки:\\
$\alpha \leq \beta \Leftrightarrow  \alpha \vdash \beta$\\
$\alpha \sim \beta \Leftrightarrow \alpha \vdash \beta$ и $\beta \vdash \alpha$\\
Определим операции и $0$, $1$:\\
$0$ -- $\alpha \& \neg \alpha = \perp$\\
$1$ -- $\alpha \rightarrow \alpha = T$\\
$\alpha \& \beta = \alpha * \beta$\\
$\alpha \vee \beta = \alpha + \beta$\\
$\neg \alpha = -\alpha$\\
Получившаяся алгебра называется \emph{алгеброй Линденбаума-Тарского} и является алгеброй Гейтинга, т.к. для нее выполняется аксиома $\alpha * \neg \alpha = 0$ (по определению).
\begin{lemma}
$\forall \beta \in V \perp \vdash \beta$ (Из лжи следует все)
\end{lemma}
\begin{proof}
$\alpha \& \neg \alpha \vdash \beta$\\
\begin{tabular}{lll}
(1) &$\alpha \& \neg \alpha$& Допущение\\
(2) &$\alpha \& \neg \alpha \rightarrow \alpha$& Сх. акс. 4\\
(3) &$\alpha \& \neg \alpha \rightarrow \neg \alpha$& Сх. акс. 5\\
(4) &$\alpha$& M.P. 1,2\\
(5) &$\neg \alpha$& M.P. 1,3\\
(6) &$\alpha \rightarrow \neg \alpha \rightarrow \beta$& Сх. акс. 10\\
(7) & $\neg \alpha \rightarrow \beta$& M.P. 4,6\\
(8) & $\beta$& M.P. 5,7\\
\end{tabular}
\end{proof}
\subsection{Теорема о полноте ИИВ относительно алгебры Гейтингa}
\label{sec-5-6}
Возьмем в качестве алгебры Гейтинга алгебру Линденбаума-Тарского - $\xi$. Она очевидно является моделью. 
\begin{theorem}
$\vDash \alpha \Rightarrow \vdash \alpha$
\end{theorem}
\begin{proof}
$\vDash \alpha \Rightarrow \llbracket \alpha \rrbracket ^ {\xi} = 1$\\
$\llbracket \alpha \rrbracket ^ {\xi} = 1 \Rightarrow 1 \leq \llbracket \alpha \rrbracket ^ {\xi}$ (По определению алгебры Л-Т)\\
$\beta \rightarrow \beta \vdash \alpha$ (По определению $\leq$ в алгебре Л-Т)\\
Т.к. $\beta \rightarrow \beta$ - тавтология, то и $\alpha$ - тавтология
\end{proof}
\subsection{Дизъюнктивность ИИВ}
\label{sec-5-7}
Используем алгебру Гёделя $\Gamma(A)$ ($\gamma$ - функция преобразования). Можно преобразовать любую алгебру Гейтинга, возьмем алгебру Л-Т. Алгебра Гёделя использует функцию преобразования: $\gamma(a)=b$ значит, что в алгебре $A$ элементу $a$ соответствует элемент $b$ из алгебры Гёделя. Порядок сохраняется естественным образом. Также добавим еще один элемент $\omega$ ($\gamma(1)=\omega$). Таким образом $\Gamma(A) = A \cup \lbrace \omega \rbrace$. Порядок в $\Gamma(A)$:
\begin{itemize}
\item $\forall a \in \Gamma(A) \setminus \lbrace 1 \rbrace$ $a \leq \omega$
\item $\omega \leq 1$
\end{itemize}
\begin{tabular}{lllll}
\cline{1-3}
\multicolumn{1}{|l|}{$a+b$}         & \multicolumn{1}{l|}{$b=1$} & \multicolumn{1}{l|}{$b=\gamma(v)$} &  &  \\ \cline{1-3}
\multicolumn{1}{|l|}{$a=1$}         & \multicolumn{1}{l|}{$1$}   & \multicolumn{1}{l|}{$1$}           &  &  \\ \cline{1-3}
\multicolumn{1}{|l|}{$a=\gamma(u)$} & \multicolumn{1}{l|}{$1$}   & \multicolumn{1}{l|}{$\gamma(u+v)$} &  &  \\ \cline{1-3}
                                    &                            &                                    &  & 
\end{tabular}
\begin{tabular}{lllll}
\cline{1-3}
\multicolumn{1}{|l|}{$a*b$}         & \multicolumn{1}{l|}{$b=1$}         & \multicolumn{1}{l|}{$b=\gamma(v)$} &  &  \\ \cline{1-3}
\multicolumn{1}{|l|}{$a=1$}         & \multicolumn{1}{l|}{$1$}           & \multicolumn{1}{l|}{$\gamma(a*v)$} &  &  \\ \cline{1-3}
\multicolumn{1}{|l|}{$a=\gamma(u)$} & \multicolumn{1}{l|}{$\gamma(u*b)$} & \multicolumn{1}{l|}{$\gamma(u*v)$} &  &  \\ \cline{1-3}
                                    &                                    &                                    &  & 
\end{tabular}\\
\begin{tabular}{lllll}
\cline{1-3}
\multicolumn{1}{|l|}{$a \rightarrow b$} & \multicolumn{1}{l|}{$b=1$} & \multicolumn{1}{l|}{$b=\gamma(v)$}             &  &  \\ \cline{1-3}
\multicolumn{1}{|l|}{$a=1$}             & \multicolumn{1}{l|}{$1$}   & \multicolumn{1}{l|}{$\gamma(a \rightarrow v)$} &  &  \\ \cline{1-3}
\multicolumn{1}{|l|}{$a=\gamma(u)$}     & \multicolumn{1}{l|}{$1$}   & \multicolumn{1}{l|}{$u \rightarrow v$}         &  &  \\ \cline{1-3}
                                        &                            &                                                &  & 
\end{tabular}
\begin{tabular}{lllll}
\cline{1-2}
\multicolumn{1}{|l|}{$a$} & \multicolumn{1}{l|}{$\neg a$} &  &  &  \\ \cline{1-2}
\multicolumn{1}{|l|}{$a=1$} & \multicolumn{1}{l|}{$\gamma(\neg a)$} &  &  &  \\ \cline{1-2}
\multicolumn{1}{|l|}{$a=\gamma(u)$} & \multicolumn{1}{l|}{$\neg u$} &  &  &  \\ \cline{1-2}
 &  &  &  & 
\end{tabular}
\begin{lemma}
Гёделева алгебра является Гейтинговой
\end{lemma}
\begin{proof}
Необходимо просто доказать аксиомы коммутативности, ассоциативности и поглощения.
\end{proof}

\begin{theorem}
$\vdash \alpha \vee \beta \Rightarrow$ либо $\vdash \alpha$, либо $\vdash \beta$
\end{theorem}
\begin{proof}
Возьмем $A$, построим $\Gamma(A)$. Если $\vdash \alpha \vee \beta$, то $\llbracket \alpha \vee \beta \rrbracket ^ {A} = 1$ и $\llbracket \alpha \vee \beta \rrbracket ^ {\Gamma(A)} = 1$.\\
Тогда по определению $+$ в алгебре Гёделя, $\llbracket \alpha \rrbracket ^ {\Gamma(A)} = 1$, либо $\llbracket \beta \rrbracket ^ {\Gamma(A)} = 1$. Тогда оно такое же и в алгебре Л-Т, а алгебра Л-Т полна.
\end{proof}
\subsection{Теорема Гливенко}
\label{sec-5-8}
\begin{theorem}
Если в ИВ доказуемо $\alpha$, то в ИИВ доказуемо $\neg\neg\alpha$.
\end{theorem}
\begin{proof}
Разберем все втречающиеся в изначальном доказательстве формулы
\begin{enumerate}
\item Заметим, что если в ИИВ доказуемо $\alpha$, то $\neg\neg\alpha$ так же доказуемо.

Докажем, что $\alpha \vdash \neg \neg \alpha$

\begin{tabular}{lll}
(1) &$\alpha$& Допущение\\
(2) &$\alpha \rightarrow \neg \alpha \rightarrow \alpha$& Сх. акс. 1\\
(3) &$\neg \alpha \rightarrow \alpha$& M.P. 1,2\\
(4) & $\neg \alpha \rightarrow (\neg \alpha \rightarrow \neg \alpha)$&Сх. акс. 1\\
(5) & $(\neg \alpha \rightarrow (\neg \alpha \rightarrow \neg \alpha)) \rightarrow 
  (\neg \alpha \rightarrow ((\neg \alpha \rightarrow \neg \alpha) \rightarrow \neg \alpha)) \rightarrow
  (\neg \alpha \rightarrow \neg \alpha)$&Сх. акс. 2\\
(6) & $(\neg \alpha \rightarrow ((\neg \alpha \rightarrow \neg \alpha) \rightarrow \neg \alpha)) \rightarrow
  (\neg \alpha \rightarrow \neg \alpha)$&M.P. 4,5\\
(7) & $(\neg \alpha \rightarrow ((\neg \alpha \rightarrow \neg \alpha) \rightarrow \neg \alpha))$ & Сх. акс. 1\\
(8) & $\neg \alpha \rightarrow \neg \alpha$ & M.P. 7,6\\
(9) & $(\neg \alpha \rightarrow \alpha) \rightarrow (\neg \alpha \rightarrow \neg \alpha) \rightarrow \neg \neg \alpha$& Сх. акс. 9\\
(10) & $(\neg \alpha \rightarrow \neg \alpha) \rightarrow \neg \neg \alpha$& M.P. 3,9\\
(11) & $\neg \neg \alpha$& M.P. 8,10\\
\end{tabular}

Значит, если $\alpha$ -- аксиома с 1-ой по 9-ую, то $\neg \neg \alpha$ также может быть доказано
\item Пусть $\alpha$ получилось по 10-ой аксиоме $\neg \neg \alpha \rightarrow \alpha$. Докажем, что $\vdash \neg \neg (\neg \neg \alpha \rightarrow \alpha)$

\begin{tabular}{lll}
(1) &$\alpha \rightarrow \neg \neg \alpha \rightarrow \alpha$& Сх. акс. 1\\
(2) &$\neg (\neg \neg \alpha \rightarrow \alpha) \rightarrow \neg \alpha$& Контрпозиция\\
(3) &$\neg \alpha \rightarrow \neg \neg \alpha \rightarrow \alpha$& Сх. акс. 10\\
(4) &$\neg (\neg \neg \alpha \rightarrow \alpha) \rightarrow \neg \neg \alpha$& Контрпозиция\\
(5) &$(\neg ( \neg \neg \alpha \rightarrow \alpha) \rightarrow \neg \alpha) \rightarrow (\neg (\neg \neg \alpha \rightarrow \alpha) \rightarrow \neg \neg \alpha) \rightarrow \neg \neg (\neg \neg \alpha \rightarrow \alpha)$& Сх. акс. 9\\
(6) &$(\neg (\neg \neg \alpha \rightarrow \alpha) \rightarrow \neg \neg \alpha) \rightarrow \neg \neg (\neg \neg \alpha \rightarrow \alpha)$& M.P. 2,5\\
(7) &$\neg \neg (\neg \neg \alpha \rightarrow \alpha)$& M.P. 4,6\\
\end{tabular}
\item Приведем конструктивное доказательство:
\begin{itemize}
\item Если $\alpha$ - аксиома, то $\neg \neg \alpha$ доказывается с помощью 1-го и 2-го пунктов
\item Если был применен M.P., то в изначальном доказательстве были $\alpha$, $\alpha \rightarrow \beta$, $\beta$. По индукционному предположению мы знаем, что $\neg \neg \alpha$, $\neg \neg (\alpha \rightarrow \beta)$. Нужно доказать $\neg \neg \beta$.\\
Давайте для начала докажем, что \[\neg \neg \alpha, \neg \neg (\alpha \rightarrow \beta), \neg \beta, \alpha, \alpha \rightarrow \beta \vdash \beta\].\\
\begin{tabular}{lll}
(1) &$\alpha$& Допущение\\
(2) &$\alpha \rightarrow \beta$& Допущение\\
(3) &$\beta$& M.P. 1,2\\
\end{tabular}\\
Значит мы знаем, что $\neg \neg \alpha, \neg \neg (\alpha \rightarrow \beta), \neg \beta, \alpha \vdash (\alpha \rightarrow \beta) \rightarrow \beta$. Теперь докажем, что \[\neg \neg \alpha, \neg \neg (\alpha \rightarrow \beta), \neg \beta, \alpha, (\alpha \rightarrow \beta) \rightarrow \beta \vdash \neg (\alpha \rightarrow \beta.)\]\\
\begin{tabular}{lll}
(1) &$((\alpha \rightarrow \beta) \rightarrow \beta) \rightarrow ((\alpha \rightarrow \beta) \rightarrow \neg \beta) \rightarrow \neg(\alpha \rightarrow \beta)$& Сх. акс. 9\\
(2) &$((\alpha \rightarrow \beta) \rightarrow \beta)$& Допущение\\
(3) &$\neg \beta \rightarrow (\alpha \rightarrow \beta) \rightarrow \neg \beta$& Сх. акс. 1\\
(4) &$\neg \beta$& Допущение\\
(5) &$(\alpha \rightarrow \beta) \rightarrow \neg \beta$& M.P. 4,3\\
(6) &$((\alpha \rightarrow \beta) \rightarrow \neg \beta) \rightarrow \neg(\alpha \rightarrow \beta)$ M.P. 2,1\\
(7) &$\neg(\alpha \rightarrow \beta)$ M.P. 5,6\\
\end{tabular}\\
Теперь мы знаем, что $\neg \neg \alpha, \neg \neg (\alpha \rightarrow \beta), \neg \beta \vdash \alpha \rightarrow \neg (\alpha \rightarrow \beta)$. Докажем, что \[ \neg \neg \alpha, \neg \neg (\alpha \rightarrow \beta), \neg \beta, \alpha \rightarrow \neg (\alpha \rightarrow \beta) \vdash \neg \alpha.\] \\
\begin{tabular}{lll}
(1) &$(\alpha \rightarrow \neg (\alpha \rightarrow \beta)) \rightarrow (\alpha \rightarrow \neg \neg (\alpha \rightarrow \beta)) \rightarrow \neg \alpha$& Сх. акс. 9\\
(2) &$\alpha \rightarrow \neg (\alpha \rightarrow \beta)$&Допущение\\
(3) &$\neg \neg (\alpha \rightarrow \beta) \rightarrow \alpha \rightarrow \neg \neg (\alpha \rightarrow \beta)$& Сх. акс. 1\\
(4) &$\neg \neg (\alpha \rightarrow \beta)$& Допущение\\
(5) &$\alpha \rightarrow \neg \neg (\alpha \rightarrow \beta)$& M.P. 4,3\\
(6) &$(\alpha \rightarrow \neg \neg (\alpha \rightarrow \beta)) \rightarrow \neg \alpha$& M.P. 2,1\\
(7) &$\neg \alpha$& M.P.5,6\\
\end{tabular}\\
Теперь мы знаем, что $\neg \neg \alpha, \neg \neg (\alpha \rightarrow \beta) \vdash \neg \beta \rightarrow \neg \alpha$. Наконец докажем, что \[ \neg \neg \alpha, \neg \neg (\alpha \rightarrow \beta), \neg \beta \rightarrow \neg \alpha \vdash \neg \neg \beta\] \\
\begin{tabular}{lll}
(1) &$(\neg \beta \rightarrow \neg \alpha) \rightarrow (\neg \beta \rightarrow \neg \neg \alpha) \rightarrow \neg \neg \beta$& Сх. акс. 9\\
(2) &$\neg \beta \rightarrow \neg \alpha$& Допущение\\
(3) &$\neg \neg \alpha \rightarrow \neg \beta \rightarrow \neg \neg \alpha$& Сх. акс. 1\\
(4) &$\neg \neg \alpha$& Допущение\\
(5) &$\neg \beta \rightarrow \neg \neg \alpha$& M.P. 4,3\\
(6) &$(\neg \beta \rightarrow \neg \neg \alpha) \rightarrow \neg \neg \beta$& M.P. 2,1\\
(7) &$\neg \neg \beta$& M.P. 5,6\\
\end{tabular}
\end{itemize}
\end{enumerate}
\end{proof}
\subsection{Топологическая интерпретация}
\label{sec-5-9}
Булеву алгебру и алгебру Гейтинга можно интерпретировать на множестве $\mathbb R^n$. Тогда заключения о общезначимости формулы можно делать более наглядно. Давайте возьмем в качестве множества алгебры все открытые подмножества $\mathbb R^n$. Определим операции следующим образом:
\begin{itemize}
\item $\alpha + \beta = \alpha \cup \beta$
\item $\alpha * \beta = \alpha \cap \beta$
\item $\alpha \rightarrow \beta = Int(\alpha^c\cup\beta)$
\item $-\alpha = Int(\alpha^c)$
\item $0 = \emptyset$
\item $1 = \cup \lbrace V \subset L \rbrace$
\end{itemize}
