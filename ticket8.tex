\section{Ticket 8: рекурс, Аккерман}
\label{sec-10}
\subsection{Рекурсивные функции}
\label{sec-10-1}
Рассмотрим примитивы, из которых будем собирать выражения:

\begin{enumerate}
\item $Z \colon \mathbb{N} \rightarrow \mathbb{N}$, $Z(x) = 0$
\item $N \colon \mathbb{N} \rightarrow \mathbb{N}$, $N(x) = x'$
\item Проекция. $U^n_i \colon \mathbb{N}^n \rightarrow \mathbb{N}$, $U^n_i (x_1, \dotsc, x_n) = x_i$
\item Подстановка. Если $f \colon \mathbb{N}^n \rightarrow \mathbb{N}$ и $g_1, \dotsc, g_n \colon \mathbb{N}^m \rightarrow \mathbb{N}$, 
    то $S\template{f,g_1,\dotsc, g_n} \colon \mathbb{N}^m \rightarrow \mathbb{N}$.

    При этом $S\langle{}f,g_1,\dotsc, g_n\rangle (x_1,\dotsc, x_m) = f(g_1(x_1,\dotsc, x_m), \dotsc, g_n(x_1,\dotsc, x_m))$
\item Примитивная рекурсия. Если $f \colon \mathbb{N}^n     \to \mathbb{N}$ и 
                                 $g \colon \mathbb{N}^{n+2} \to \mathbb{N}$, то
	$R\ltemplate f, g\rtemplate(x_1\ldots{}x_n, n) = \begin{cases}
    	f(x_1, \dotsc, x_n) & n = 0 \\
    	g(x_1, \dotsc, x_n, n, R\ltemplate f, g\rtemplate(x_1, \dotsc, x_n, n - 1)) & n > 0
	\end{cases}$
\item Минимизация. Если $f \colon \mathbb{N}^{n+1} \rightarrow \mathbb{N}$,
    то $\mu \template{f}\colon \mathbb{N}^n \rightarrow \mathbb{N}$, при этом
  $\mu \langle{}f\rangle (x_1,\dotsc,x_n)$ --- такое минимальное число $y$, что $f(x_1,\dotsc, x_n, y) = 0$.
  Если такого $y$ нет, результат данного примитива неопределен.
\end{enumerate}

Пример:
\[a + b = R\template{U^2_1, S\template{N, U^3_3}}(a, b)\]
\subsection{Характеристическая функция и рекурсивное отношение}
\label{sec-10-2}
\begin{itemize}
\item \emph{Характеристическая фукнция} -- функция от выражения, которая возвращает $1$ если выражение истинно, $0$ иначе.
\item \emph{Рекурсивное отношение} -- отношение, характеристическая функция
которого рекурсивна.
\end{itemize}
\subsection{Аккерман не примитивно-рекурсивен, но рекурсивен (второе)}
\label{sec-10-3}
Функция Аккермана -- это функция, удовлетворяющая следующим правилам:
\[
    A(m,n) = \begin{cases}
        n+1 & m = 0\\
        A(m-1,n) & m > 0, n = 0\\
        A(m-1,A(m,n-1)) & m > 0, n > 0
    \end{cases}
\]
Например:
\[A(2, 0) = A(1, 1) = A(0, A(1, 0)) = A(0, 2) = 3\]
\begin{lemma}
$A(m, n) \geq 1$
\end{lemma}
\begin{proof}
$A(m, n)$ определена только на натуральных числах\\
$A(0, 0) = 1, A(1, 0) = A(0, 1) = 2, A(0, 1) = 2$, а все остальное ещё больше
\end{proof}
\begin{lemma}
\label{lemma1a}
$A(1, n) = n + 2$
\end{lemma}
\begin{proof}
$A(1, n) = A(0, A(1, n - 1))$
$= A(0, A(0, A(1, n - 2)))$\\
$= A(0, A(0, A(0, \ldots{} A(1, 0))))$
$= A(0, A(0, A(0, \ldots{} 2)))$\\
$= n + 2$ $ ($n$ раз инкрементируем двойку)$\\
\end{proof}
\begin{lemma}
\label{lemma1b}
$A(2, n) = 2n + 3$
\end{lemma}
\begin{proof}
$A(2, n)
= A(1, A(1, \ldots{} A(2, 0)))
= A(1, A(1, \ldots{} 3))
= 2n + 3$ ($n$ раз к тройке прибавляем $A(0, 1) = 2$)
\end{proof}
\begin{lemma}
\label{lemma2}
$A(m, n) \geq n + 1$
\end{lemma}
\begin{proof}
В первом случае $A \geq n + 1 = n + 1$\\
Во втором $A$ может перейти в первый случай, который работает
хорошо, или в третий.\\
В третьем случае мы можем получить $A(0, n)$ если первый аргумент
был нулем, тогда все ок, можем получить $A(1, 0)$, тогда это второй
случай, для него условие выполнено.\\
Третий ссылается на второй, а второй на третий, но тут
нет противоречия, потому что мы знаем, что функция Аккермана
завершается.
\end{proof}
\begin{lemma}
\label{lemma3a}
$A(m, n) < A(m, n + 1)$
\end{lemma}
\begin{proof}
Проведем индукцию по $m$:
\begin{itemize}
\item База:\\
$A(0, n) = n + 1 < n + 2 = A(0, n + 1)$
\item Переход:\\
$A(k + 1, m) < A(k + 1, m) + 1$\\
$\leq A(k, A(k + 1, m))$ (По \ref{lemma2})\\
$\leq A(k + 1, m + 1)$   (3-е свойства ф-ии Аккермана)
\end{itemize}
\end{proof}
\begin{lemma}
\label{lemma3b}
$A(m, n + 1) \leq A(m + 1, n)$
\end{lemma}
\begin{proof}
Проведем индукцию по $n$:
\begin{itemize}
\item База:\\
$A(m, 0 + 1) = A(m, 1) = A(m + 1, 0)$ (ii)
\item Переход, предположение:
    \begin{align*}
        A(m, j + 1) &\leq A(m + 1, j) \tag{По \ref{lemma2}}\\
        (j + 1) + 1 &\leq A(m, j + 1)\\
        A(m, (j + 1) + 1) &\leq A(m, A(m, j + 1)) \tag{По монотонности}\\
        A(m, A(m, j + 1)) &\leq A(m, A(m + 1, j)) \tag{По монотонности + предположение}\\
        A(m, (j + 1) + 1) &\leq A(m, A(m + 1, j)) = A(m + 1, j + 1) \tag{3-е свойство ф-ии Аккермана}
    \end{align*}
\end{itemize}
\end{proof}
\begin{lemma}
\label{lemma3c}
$A(m, n) < A(m + 1, n)$
\end{lemma}
\begin{proof}
$A(m, n) < A(m, n + 1) \leq A(m + 1, n)$ (По \ref{lemma3a}, \ref{lemma3b})
\end{proof}
\begin{lemma}
\label{lemma4}
$A(m_1, n) + A(m_2, n) < A(\max(m_1, m_2) + 4, n)$
\end{lemma}
\begin{proof}
\begin{align*}
&A(m_1, n) + A(m_2, n)\\
    \le {}&A(\max(m_1, m_2), n) + A(\max(m_1, m_2), n)\\
    = {}&2 \cdot A(\max(m_1, m_2), n)\\
    < {}&2 \cdot A(\max(m_1, m_2), n) + 3\\
    = {}&A(2, A(\max(m_1, m_2), n)) \tag{По \ref{lemma1a}}\\
    < {}&A(2, A(\max(m_1, m_2) + 3, n)) \tag{Строгая монотоннасть по обоим арг.}\\
    < {}&A(\max(m_1, m_2) + 2, A(\max(m_1, m_2) + 3, n)) \tag{По \ref{lemma3c}}\\
    = {}&A(\max(m_1, m_2) + 3, n + 1)   \tag{3-е свойства ф-ии Аккермана}\\
    \le {}&A(\max(m_1, m_2) + 4, n) \tag{По \ref{lemma3b}}\\
\end{align*}
\end{proof}
\begin{lemma}
\label{lemma5}
$A(m, n) + n < A(m + 4, n)$
\end{lemma}
\begin{proof}
    \begin{align*}
    A&(m, n) + n \\
    < A&(m, n) + n + 1 \\
    = A&(n, m) + A(0, n) \\
    < A&(m + 4, n)
    \end{align*}
\end{proof}
\begin{theorem}
Функция аккерманна не притивно-рекурсивна
\end{theorem}
\begin{proof}
TODO
\end{proof}
\iffalse
\subsubsection{Аккерманн не примитивно-рекурсивен}
\label{sec-10-3-10}
Пусть f(n_1\dotsc, n_k) - примитивная рекурсинвная функция, k ≥ 0.
\exists J:f(n_1,\dotsc,n_k)<A(J, ∑(n_1,\dotsc, n_k))

]n\textasciitilde{} = (n_1\dotsc, n_k)
Индукция по рекурсивным функциям
\begin{itemize}
\item База:
f(n\textasciitilde{}) - N или Z или Uₖⱼ
\begin{enumerate}
\item f(n\textasciitilde{}) = N, k = 1; Пусть J=1, по (i) и лемме 3c
f(n) = N(n) = n + 1 = A(0, n) < A(1, n) = A(J, n) = A(J, ∑(n\textasciitilde{}))
\item f(n\textasciitilde{}) = Z, k = 1;
f(n) = 0 < A(J, n) (потому что A ≥ 1) = A(J, ∑(n\textasciitilde{}))
\item f(n\textasciitilde{}) = Uₖⱼ; k = k; Пусть J=1
f(n_1\dotsc, n_k) = Uₖⱼ(n_1\dotsc, n_k) = nⱼ
Пусть nⱼ = 0, тогда f(n) = 0 < A(J, ∑(n\textasciitilde{})) для любого нормального J
Пусть nⱼ > 0, тогда f(n) = (nⱼ - 1) + 1 = A(0, nⱼ - 1) < A(1, n)
= A(J, ∑(n\textasciitilde{}))
\end{enumerate}
\item Переход
\begin{enumerate}
\item Предположим, что f(n\textasciitilde{}) = S<h, g_1\ldots{}g_m>(n\textasciitilde{}) = h(g_1(n\textasciitilde{}),\ldots{}g_m(n\textasciitilde{}))
По предположению индукции существует J₀ для h, J_1\ldots{}J_m для g_1\ldots{}g_m.
f(n\textasciitilde{}) = h(g_1(n\textasciitilde{}),..)
< A(J₀, ∑\{i=1..m\}(n\textasciitilde{}))            по выбору J₀
< A(J₀, ∑(A(Jᵢ, ∑(n\textasciitilde{}))))           по выбору Jᵢ и строгой монотонности
// J* = \max(J_1..J_m) + 4(m - 1)
< A(J₀, A(J*, ∑(n\textasciitilde{})))             по лемме 4 примененной m-1 раз
< A(J₀, A(J*+1, ∑(n\textasciitilde{})))           по монотонности
≤ A(J₀, A(\max(J₀, J*) + 1, ∑(n\textasciitilde{}))) по монотонности
≤ A(\max(J₀, J*) + 1, ∑(n\textasciitilde{}) + 1)   (iii)
= A(\max(J₀, J*) + 2, ∑(n\textasciitilde{}))       по лемме 3b
Тогда пусть j=\max(J₀, J*) + 2
\item Пусть f(n\textasciitilde{}) = R<h,g>(n\textasciitilde{})
f(n_1, \dotsc, n_k, 0) = h(n_1, \dotsc, n_k)
f(n_1, \dotsc, n_k, m+1) = g(n_1, \dotsc, n_k, m, f(n_1, \dotsc, n_k, m))
По предположению имеем J₀ (h), J_1 (g).
] J = \max(J₀, J_1) + 4
\begin{enumerate}
\item f(n\textasciitilde{}, 0)
≤ f(n\textasciitilde{}, 0) + ∑(n\textasciitilde{})
= h(n\textasciitilde{}) + ∑(n\textasciitilde{})
< A(J₀, ∑(n\textasciitilde{})) + ∑(n\textasciitilde{})
< A(J₀ + 4, ∑(n\textasciitilde{}))                   по лемме 5
< A(J, ∑(n\textasciitilde{}))                       по монотонности
= A(J, ∑(n\textasciitilde{}) + 0)
\item f(n\textasciitilde{}, k + 1)
= g(n\textasciitilde{}, k, f(n\textasciitilde{}, k))
< A(J_1, ∑(n\textasciitilde{}) + k + f(n\textasciitilde{}, k))        по выбору J_1
< A(J_1, ∑(n\textasciitilde{}) + k + 1 + f(n\textasciitilde{}, k))    по монотонности
= A(J_1, A(0, ∑(n\textasciitilde{}) + k) + f(n\textasciitilde{}, k))  (i)
< A(J_1, A(0, ∑(n\textasciitilde{}) + k) + H(J, ∑(n\textasciitilde{})+k)) по предположению
< A(J_1, A(J, ∑(n\textasciitilde{})+k)+A(J, ∑(n\textasciitilde{}) + k)) по монотонности (J > 0)
= A(J_1, 2 * [A(J, ∑(n\textasciitilde{}) + k)])
< A(J_1, 2 * [A(J, ∑(n\textasciitilde{}) + k)] + 3)
= A(J_1, A(2, A(J, ∑(n\textasciitilde{}) + k)))        по лемме 1
< A(J_1, A(J_1 + 1, A(J, ∑(n\textasciitilde{}) + k)))   по строгой монотонности (J_1 > 2)
= A(J_1 + 1, A(J, ∑(n\textasciitilde{}) + k) + 1)      (iii)
≤ A(J_1 + 2, A(J, ∑(n\textasciitilde{}) + k))
< A(J - 1, A(J, ∑(n\textasciitilde{}) + k))           по монот. J > \max(..) + 4
= A(J, ∑(n\textasciitilde{}) + (k + 1))               (iii), J != 0
\end{enumerate}
\end{enumerate}
\end{itemize}
\fi
\begin{theorem}
Функция Аккермана рекурсивна
\end{theorem}
\begin{proof}
Можем сказать, что он рекурсивный, потому что мы можем
его написать на компуктере, а тьюринг выражается в рекурсивных функциях.
\end{proof}
