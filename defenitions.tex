\section{Определения (нужно знать идеально)}
\label{sec-2}
Определения тут зачастую дублируют то, что написано в самом
конспекте, поэтому удаление этого блока сэкономит бумагу при
печати.
\subsection{ИВ}
\label{sec-2-1}
Формальная система с алгеброй Яськовского $J_{0}$ в качестве модели, множество истинностных значений $\lbrace0, 1\rbrace$.
Формальная теория нулевого порядка, кванторов нету, предикаты -- это пропозициональные переменные.
Аксиомы:
\begin{enumerate}
\item $\alpha \to \beta \to \alpha$
\item $(\alpha \to \beta) \to (\alpha \to \beta \to \gamma) \to (\alpha \to \gamma)$
\item $\alpha \to \beta \to \alpha \& \beta$
\item $\alpha \& \beta \to \alpha$
\item $\alpha \& \beta \to \beta$
\item $\alpha \to \alpha \lor \beta$
\item $\beta \to \alpha \lor \beta$
\item $(\alpha \to \beta) \to (\gamma \to \beta) \to (\alpha \lor \gamma \to \beta)$
\item $(\alpha \to \beta) \to (\alpha \to \lnot \beta) \to \lnot \alpha$
\item $\lnot \lnot \alpha \to \alpha$
\end{enumerate}
\subsection{Общезначимость, доказуемость, выводимость}
\label{sec-2-2}
\begin{itemize}
\item Общезначимость формулы -- ее свойство в теории с моделью.
Общезначимость можно определить как угодно, в принципе.
Например в ИВ общезначимость -- это что оценка формулы на любых значениях свободных переменных отображает в 1.
В модели крипке -- существование формулы во всех мирах и т.д.
\item Доказуемость -- свойство формулы в теории, значащее, что существует
доказательство для этой формулы. Доказательство для теории тоже определяется
по разному (последовательность утверждений, каждое из которых есть аксиома
или следует по правилу вывода из предыдущих в ИВ, дерево с выводами в $S_{\infty}$)
\item Выводимость -- в общем случае часто используется как аналог доказуемости,
в ИВ это доказуемость из всего, что и ранее + из посылок.
\end{itemize}
\subsection{Теорема о дедукции для ИВ}
\label{sec-2-3}
Теорема, утверждающая, что из $\Gamma, \alpha \vdash \beta$ следует $\Gamma \vdash \alpha \to \beta$ и наоборот.\\
Доказывается вправо поформульным преобразованием, влево
добавлением 1 формулы. Работает в ИВ, ИИВ, предикатах.
\subsection{Теорема о полноте исчисления высказываний}
\label{sec-2-4}
\theorem[о полноте исчисления высказываний]{
Исчисление предикатов полно}.\\
Общий ход д-ва: строим док-ва для конкретных наборов перменных,
$2^n$, где $n$ -- количество возможных переменных. Потом их мерджим.
\subsection{ИИВ}
\label{sec-2-5}
Берем ИВ, выкидываем 10 аксиому, добавляем $\alpha \rightarrow \neg \alpha \rightarrow \beta$.\\
Она доказывается и в ИВ:
\begin{lemma}
$\alpha, \alpha \vee \neg \alpha, \neg \alpha \vdash \beta$
\end{lemma}
\begin{tabular}{lll}
(1) & $\alpha$& Допущение\\
(2) & $\neg \alpha$& Допущение\\
(3) & $\alpha \rightarrow \neg \beta \rightarrow \alpha$& Сх. акс. 1\\
(4) & $\neg \beta \rightarrow \alpha$& M.P. 1,3\\
(5) & $\neg \alpha \rightarrow \neg \beta \rightarrow \neg \alpha$& Сх. акс. 1\\
(6) & $\neg \beta \rightarrow \neg \alpha$& M.P. 2,5\\
(7) & $(\neg \beta \rightarrow \alpha) \rightarrow (\neg \beta \rightarrow \neg \alpha) \rightarrow (\neg \neg \beta)$& Сх. акс. 9\\
(8) & $(\neg \beta \rightarrow \neg \alpha) \rightarrow (\neg \neg \beta)$& M.P. 4,7\\
(9) & $\neg \neg \beta$& M.P. 6,8\\
(10) & $\neg \neg \beta \rightarrow \beta$& Сх. акс. 10\\
(11) & $\beta$& M.P. 9,10\\
\end{tabular}\\
А еще в ИИВ главная фишка -- недоказуемо $\alpha \lor \lnot \alpha$ (можно подобрать такую модель).
\subsection{Теорема Гливенко}
\label{sec-2-6}
\theorem{Гливенко}
Если в ИВ доказуемо $\alpha$, то в ИИВ доказуемо $\lnot \lnot \alpha$

Общий ход д-ва: говорим, что если в ИИВ доказуема $\delta_{i}$,
то в ней же доказуема $\lnot \lnot \delta_{i}$. Доказываем руками двойное
отрицание 10 аксиомы и то же самое для MP.
\subsection{Порядки}
\label{sec-2-7}
\begin{definition}
    Частичный порядок – рефлексивное, антисимметричное, транзитивное
отношение.
\end{definition}
\begin{definition}
    Частично упор. мн-во -- множество с частичным порядком на элементах.
\end{definition}
\begin{definition}
    Линейно упорядоч. мн-во -- множество с частичным порядком, в котором
    два любых элемента сравнимы.
\end{definition}
\begin{definition}
    Фундированное мн-во -- частично упорядоч. множество, в котором каждое
    непустое подмножество имеет минимальный элемент.
\end{definition}
\begin{definition}
    Вполне упорядоченное множество -- фундированное множество с линейным
    порядком.
\end{definition}
\subsection{Решетки (все свойства)}
\label{sec-2-8}
\begin{itemize}
\item Просто Решетка -- это $(L, +, *)$ в алгебраическом смысле и $(L, \le)$ в порядковом.
Решетку можно определить как алгебраическую структуру через
аксиомы: коммутативность, ассоциативность, поглощение.
Решетку можно определить как упорядоченное множество через
множество с частичным порядком на нем, тогда операции +, * определяются
как $\sup$ и $\inf$:
\begin{align*}
    \sup p &= \min \{u \mid u \ge all s \in p\} \\
    \inf p &= \max \{u \mid u \le all s \in p\} \\
    a + b  &= \sup \{a, b\} \\
    a * b  &= \inf \{a, b\}
\end{align*}
Если для двух элементов всегда можно определить $a + b$ и $a * b$, то такое
множество назывется решеткой.
\item Дистрибутивная решетка -- решетка, в которой работает дистрибутивность:
$a * (b + c) = (a * b) + (b * c)$
\item Импликативная решетка -- всегда существует псевдодополнение b ($b \to a$)
$a \to b = \max \lbrace c | c \times a \le b \rbrace$
Имеет свойствa, что в ней всегда есть максимальный элемент $a \to a$ и что
она дистрибутивна.
\end{itemize}
\subsection{Булевы/псевдобулевы алгебры}
\label{sec-2-9}
\begin{itemize}
\item Булева алгебра можно определить так:
\begin{enumerate}
\item $(L, +, *, -, 0, 1)$ с выполненными аксиомами -- коммутативность, ассоциативность,
    поглощение, две дистрибутивности и $a * -a = 0$,
$a + -a = 1$.
\item Импликативная решетка над фундированным множеством.

Тогда мы в ней определим $1$ как $a \to a$ (традиционно для импликативной),
отрицание как $-a = a \to 0$, и тогда последняя аксиома из
предыдущего определения будет свойством:
\[a * -a = a * (a \to 0) = a * (\max c: c * a \le 0) = a * 0 = 0\]
Насчет второй аксиомы -- должно быть 1. То есть лучше как-то
через аксиомы определять, видимо.
\[a + -a = a + (a \to 0) = a + (\max c: c * a \le 0) = a + 0 = a\] // не 1
\end{enumerate}
\item Псевдобулева алгебра -- это импликативная решетка над фундированным
множеством с $\lnot a = (a \to 0)$
\end{itemize}
\subsection{Топологическая интерпретация ИИВ}
\label{sec-2-10}
Булеву алгебру и алгебру Гейтинга можно интерпретировать
на множестве $\mathbb{R}^{n}$. Тогда заключения о общезначимости формулы
можно делать более наглядно.
Давайте возьмем в качестве множества алгебры все открытые
подмножества $\mathbb{R}^{n}$. Определим операции следующим образом:
\begin{enumerate}
\item $a + b \defeq a \cup b$
\item $a * b \defeq a \cap b$
\item $a \to b \defeq Int(a^c \cup b)$
\item $-a \defeq \Int(a^c)$
\item $0 \defeq \emptyset$
\item $1 \defeq \bigcup\{\text{всех мн-в в $L$}\}$
\end{enumerate}
\subsection{Модель Kрипке}
\label{sec-2-11}
$Var = \{P, Q, \dotsc\}$
Модель Крипке – это $\langle W, \leq, v\rangle$, где
\begin{itemize}
\item $W$ -- множество <<миров>>
\item $\leq$ -- частичный порядок на W (отношение достижимости)
\item $v \colon W \times Var \to \{0, 1, \_\}$ -- оценка перменных на $W$, монотонна
(если $v(x, P) = 1$, $x \leq y$, то $v(y, P) = 1$ -- формулу нельзя un'вынудить)
\end{itemize}

Правила:
\begin{itemize}
    \item $W, x \vDash  P \Leftrightarrow v(x, P) = 1 \text{, если $P \in Var$}$
\item $W, x \vDash  (A \& B) \Leftrightarrow W, x \vDash  A \& W, x \vDash  B$
\item $W, x \vDash  (A \lor B) \Leftrightarrow W, x \vDash  A \lor W, x \vDash  B$
\item $W, x \vDash  (A \to B) \Leftrightarrow \forall  y \ge x (W, y \vDash  A ⇒ W, y \vDash  B)$
\item $W, x \vDash  \lnot A \Leftrightarrow \forall  y \in x (W, x \lnot \vDash  A)$
\end{itemize}

В мире разрешается быть не вынужденной переменной и ее отрицанию
одновремеменно.
Формула называется тавтологией в ИИВ с моделью Крипке, если она
истинна (вынуждена) в любом мире любой модели Крипке.
\subsection{Вложение Крипке в алгебры Гейтинга}
\label{sec-2-12}
Возьмем модель Крипке, возьмем какое-то объединение поддеревьев
со всеми потомками, каждое такое объединение пусть будет входить
в алгебру Гейтинга. $\le$ -- отношение <<быть подмножеством>>.
Определим $0$ как $\emptyset$ (пустое объединение поддеревьев);
Определим операции:
\begin{align*}
    + &= \cup,\\
    * &= \cap,\\
    a \to b &= \bigcup \{z \in H \mid z \le x^c \cup  y\}
\end{align*}
Так созданное множество с операциями является импликативной
решеткой, в которой мы определим $-a = a \to 0$, получим булеву алгебру.
\subsection{Полнота ИИВ в алгебрах Гейтинга и моделях Крипке}
\label{sec-2-13}
ИИВ полно относительно алгебр Гейтинга и моделей Крипке.
Общий ход доказательства первого сводится к вложению
в Гейтинга алгебры Линденбаума-Тарского, а второго -
к построению дизъюнктивного множества всех доказуемых
формул, являющегося миром Крипке.
\subsection{Нетабличность ИИВ}
\label{sec-2-14}
Не существует полной модели, которая может быть выражена таблицей
(конечной -- алгебра Гейтинга и Крипке не табличны, так как и там и
там связки определяются иначе).
От противного соорудим табличную модель и покажем, что она не полна,
привев пример большой дизъюнкции из импликаций, для которой можно
построить модель Крипке в которой она не общезначима.
\subsection{Предикаты}
\label{sec-2-15}
Теория первого порядка, расширяющая исчисление высказываний.
Добавляются две новые аксиомы
$\forall x.A \to A[x:=\eta]$, где $\eta$ свободна для подстановки в A
$A[x:=\eta] \to \exists x.A, -//-$

Правила вывода:
\[\infer{A \to \forall x . B}{A \to B}\]
 x не входит сводобно в А

\[\infer{\exists x . A \to B}{A \to B}\]
 x не входит свободно в В
\subsection{Теорема о дедукции в предикатах}
\label{sec-2-16}
Аналогично 1 теореме о дедукции в ИВ, но в доказательстве должны
отсутствовать применения правил для кванторов по переменным входящих
свободно в выражение $\gamma$
$\Gamma, \gamma \vdash a \Rightarrow \Gamma \vdash \gamma \to a$
\subsection{Теорема о полноте исчисления предикатов}
\label{sec-2-17}
Исчисление предикатов полно (заметим, что относительно любой модели).
Суть в том, что если предикаты непротиворечивы, то у них есть модель.
Если у них есть модель, то типа там можно по контрпозиции показать $\vDash a$.
\subsection{Теории первого порядка, определение структуры и модели}
\label{sec-2-18}
Теория первого порядка -- это формальная система с кванторами по
функциональным символам, но не по предикатам. Рукомахательное
определение – это фс с логикой первого порядка в основе, в которой
абстрактные предикаты и функциональные символы определяются точно
(а может такое определение даже лучше).

Структура по ДГ:\\
Структурой теории первого порядка мы назовем упорядоченную тройку
$\langle D, F, P\rangle$, где $F$ — списки оценок для 0-местных, 1-местных и т.д.
функций, и $P$ = $P_{0}, P_{1} \ldots$ — списки оценок для 0-местных,
1-местных и т.д. предикатов, $D$ — предметное множество.

Понятие структуры — развитие понятия оценки из исчисления предикатов.
Но оно касается только нелогических составляющих теории; истинностные
значения и оценки для связок по-прежнему определяются исчислением
предикатов, лежащим в основе теории. Для получения оценки формулы
нам нужно задать структуру, значения всех свободных индивидных
переменных, и (естественным образом) вычислить результат.

Структура по-моему:\\
Все то же самое определение из ИВ. Мы просто забиваем на предикаты
в ИВ (не определям их), расширяем нашу сигнатуру (добавляя конкретные
предикаты и функциональные символы), определяем для нее интерпретацию.

И как всегда,..\\
Модель – это корректная структура (любое доказуемое утверждение должно
быть в ней общезначимо).
\subsection{Аксиоматика Пеано}
\label{sec-2-19}
Множество $N$ удовлетворяет аксиоматике Пеано, если:
\begin{enumerate}
\item $0 \in N$
\item $x \in N, succ(x) \in N$
\item $\nexists x \in N : (succ(x) = 0)$
\item $(succ(a) = c \& succ(b) = c) \to a = b$
\item $P(0) \& \forall n.(P(n) \to P(succ(n))) \to \forall n.P(n)$
\end{enumerate}
\subsection{Формальная арифметика -- аксиомы}
\label{sec-2-20}
Формальная арифметика -- это теория первого порядка, у которой
сигнатура определена как: (циферки, логические связки, алгебр.
связки, '), а интерпретацию сейчас будем определять.
Интерпретация определяет два множества -- $V, P$ -- истинностные и
предметные значения. Пусть множество $V = \lbrace 0, 1 \rbrace$ по-прежнему.
P = \{всякие штуки, которые мы можем получать из логических связок и 0\}

Определим оценки логических связок естественным образом.

Определим алгебраические связки так:
\begin{align*}
    +(a, 0) &= a \\
    +(a, b')& = (a + b)' \\
    *(a, 0) &= 0 \\
    *(a, b')& = a * b + a \\
\end{align*}
\subsubsection{Аксиомы}
\label{sec-2-20-1}
\begin{enumerate}
\item $a = b \to a' = b'$
\item $a = b \to a = c \to b = c$
\item $a' = b' \to a = b$
\item $\lnot (a' = 0)$
\item $a + b' = (a + b)'$
\item $a + 0 = a$
\item $a * 0 = 0$
\item $a * b' = a * b + a$
\item $\phi[x:=0] \& \forall x.(\phi \to \phi[x:=x']) \to \phi$ // $\phi$ содержит св.п x
\end{enumerate}
\subsection{Рекурсивные функции}
\label{sec-2-21}
$Z(x) = 0$\\
$N(x) = x + 1$\\
$U^n_i(x_1,\dotsc, x_n) = x_i$\\
$S\ltemplate f, g_1, \dotsc, g_n\rtemplate (x_1,\dotsc,x_m) = f(g_1(x_1\ldots{}x_m),\ldots{}g_n(x_1,\dotsc,x_m))$\\
$R\ltemplate f, g\rtemplate(x_1\ldots{}x_n, n) = \begin{cases}
    f(x_1\ldots{}x_n) & n = 0 \\
    g(x_1\ldots{}x_n, n, R\ltemplate f, g\rtemplate(x_1\ldots{}x_n, n - 1)) & n > 0
\end{cases}$\\
$\mu\ltemplate f\rtemplate(x_1, \dotsc, x_n)$ -- минимальное k, такое что $f(x_1\ldots{}x_n, k) = 0$
\subsection{Функция Аккермана}
\label{sec-2-22}
\begin{align*}
    A(0, n) &= n + 1 \\
    A(m, 0) &= A(m - 1, 1) \\
    A(m, n) &= A(m - 1, A(m, n - 1))
\end{align*}
\subsection{Существование рек.ф-й не явл. ф-ей Аккермана (определение конечной леммы)}
\label{sec-2-23}
Пусть $f(n_1,\dotsc,n_k)$ -- примитивная рекурсинвная функция, $k \ge 0$.
\[\exists J:f(n_1\ldots{}n_k)<A(J, \sum(n_1,\ldots{}n_k))\]
Доказывается индукцией по рекурсивным функциям.
\subsection{Представимость}
\label{sec-2-24}
Функция $f:N^n\to N$ называется представимой в формальной арифметике, если
существует отношение $a(x_1\ldots{}x_{n+1})$, ее представляющее, причем выполнено
следующее:
\begin{enumerate}
\item $f(a,b,\ldots{}) = x \Leftrightarrow \vdash a(\overline a, \overline b,\dotsc, \overline x$)
\item $\exists !x.f(a,b,\ldots{}x)$ (вот это свойство вроде бы не обязательно, но ДГ его писал).
\end{enumerate}
\subsection{Выразимость}
\label{sec-2-25}
Отношение n называется выразимым, если существует предикат N его
выражающий, такой что
\begin{enumerate}
    \item $n(x_1, \dotsc, x_n) истинно \Rightarrow \vdash N(\overline{x_1}, \dotsc, \overline{x_n}$)
    \item $n(x_1, \dotsc, x_n) ложно \Rightarrow \vdash \lnot N(\overline{x_1}, \dotsc, \overline{x_n}$)
\end{enumerate}
\subsection{Лемма о связи представимости и выразимости}
\label{sec-2-26}
Если $n$ выразимо, то $C_n$ представимо.
$C_n = 1$ если $n$, и нулю если $!n$
\subsection{Бета-функция Гёделя, Г-последовательность}
\label{sec-2-27}
$\beta(b, c, i) = k_i$
Функция, отображающая конечную последовательность из $N (a_i)$ в $k_i$.
Работает через магию, математику, простые числа и Гёделеву
последовательность, которая подходит под условия китайской
теоремы об остатках.
\[\beta(b, c, i) = b \% ((i + 1) * c + 1)\]
\subsection{Представимость рек.ф-й в ФА (знать формулы для самых простых)}
\label{sec-2-28}
Рекурсивные функции представимы в ФА
\begin{enumerate}
\item $z(a, b) = (a = a) \& (b = 0)$
\item $n(a, b) = (a = b')$
\item $u^n_i = (x_1 = x_1) \& \ldots{} \& (x_n = x_n) \& (x_{n+1} = x_i)$
\item $s(a_1\ldots{}a_m, b) = \exists b_1\ldots{}\exists b_n(G_1(a_1\ldots{}a_n, b_1) \& \ldots{} \& Gn(a_1\ldots{}a_m, b_n)$
\item $\begin{aligned}[t]
        r(x_1,\dotsc,x_n, k, a&) =\\
    \exists b\exists c(&\exists k(\beta(b, c, 0, k) \& \phi(x_1, \dotsc, x_n, k)) \&\\
        &B(b, c, x_{n+1}, a)\&\\
        &\forall k(k<x_{n+1} \to \exists d\exists e(B(b,c,k,d)\&B(b,c,k',e)\&G(x_1\ldots{}x_n,k,d,e))))\\
    \end{aligned}$
\item $m\ltemplate F\rtemplate(x_1,\dotsc, x_{n+1}) = F(x_1, \dotsc, x_n, x_{n+1}, 0) \& \forall y((y < x_{n+1}) \to \lnot F(x_1,\dotsc, x_n, y, 0))$
\end{enumerate}
\subsection{Гёделева нумерация (точно)}
\label{sec-2-29}
\begin{center}
\begin{tabular}{lrl}
$a$ & $\Godel{a}$ & описание\\
\hline
$($        & $3$ & \\
$)$        & $5$ & \\
$,$        & $7$ & \\
$\lnot$    & $9$ & \\
$\to$      & $11$ & \\
$\lor$     & $13$ & \\
$\&$       & $15$ & \\
$\forall$  & $17$ & \\
$\exists$  & $19$ & \\
$x_k$      & $21 + 6 \cdot k$ & переменные\\
$f^n_k$    & $23 + 6 \cdot 2^k \cdot 3^n$ & n-местные функцион. символы (', +, *)\\
$P^n_k$    & $25 + 6 \cdot 2^k \cdot 3^n$ & n-местные предикаты (=)\\
\hline
\end{tabular}
\end{center}

\subsection{Выводимость и рекурсивные функции (че там с Тьюрингом)}
\label{sec-2-30}
Основные тезисы по вопросу:
\begin{itemize}
    \item $\operatorname{Emulate}(\mathrm{input}, \mathrm{prog}) =
        \plog(R\template{f,g}(\template{`S, input, 0},  , \mathrm{pb}, \mathrm{pc},
            \mathrm{tb}, \mathrm{tc}, \operatorname{steps}(-//-)), 1) == F$
        \item $\begin{multlined}[t]
                \operatorname{Proof}(\mathrm{term}, \mathrm{proof}) = \operatorname{Emulate}(\mathrm{proof}, \mathrm{MY\_PROOFCHECKER})\\
                \&\& (\plog(\mathrm{proof}, \mathrm{len}(\mathrm{proof})) = \mathrm{term})
            \end{multlined}$
    \item Любая представимая в ФА ф-я является рекурсивной
        \begin{multline*}
            f(x_1, \dotsc, x_n) = \\
            \plog(μ\template{S\template{G_\phi, U_{n+1, 1}, \dotsc, U_{n+1, n},
            \plog(U_{n+1, n+1}, 1),
            \plog(U_{n+1, n+1}, 2)}}(x_1,\dotsc, x_n), 1)
        \end{multline*}

    $G_\phi$ тут принимает $n + 2$ аргумента: $x_1\ldots{}x_n, p, b$ и возвращает 0 если
    p -- доказательство $\phi(x_1\ldots{}xₙ, p)$, представляющего f.
\end{itemize}
\subsection{Непротиворечивость}
\label{sec-2-31}
Теория непротиворечива, если в ней нельзя одновременно
вывести $a$ и $\lnot a$.
Одновременная выводимость $\lnot a$ и $a$ эквивалентна выводимости
$a \& \lnot a$
\subsection{\texorpdfstring{$\omega$}{w}-непротиворечивость}
\label{sec-2-32}
Теория $\omega$-непротиворечива, если из $\forall \phi(x) \vdash \phi(\overline x)$ следует
$\nvdash  \exists p\lnot \phi(p)$. Проще говоря, если мы взяли
формулу, то невозможно вывести одновременно $\exists x\lnot A(x)$
и $A(0), A(1), \dotsc$
\subsection{Первая теорема Гёделя о неполноте}
\label{sec-2-33}
\begin{enumerate}
    \item Если формальная арифметика непротиворечива, то недоказуемо $\sigma(\Godel{\overline \sigma})$
    \item Если формальная арифметика $\omega$-непротиворечива, то недоказуемо $\lnot \sigma(\Godel{\overline \sigma})$
\end{enumerate}
\subsection{Первая теорема Гёделя о неполноте в форме Россера}
\label{sec-2-34}
Если формальная арифметика непротиворечива, то в ней найдется
такая формула $\phi$, что $\nvdash \phi$ и $\nvdash \lnot \phi$
\subsection{Consis}
\label{sec-2-35}
Consis -- утверждение, формально доказывающее непротиворечивость ФА

То есть $\vdash Consis => ФА$ непротиворечива
\subsection{Условия Г-Б (наизусть)}
\label{sec-2-36}
Пусть $\pi g(x, p)$ выражает $\Proof(x, p)$.
$\pi (x) = \exists t.\pi g(x, t)$ действительно показывает,
что выражение доказуемо, если
\begin{enumerate}
\item $\vdash a => \vdash \pi(\Godel{\overline a})$
\item $\vdash \pi \left(\Godel{\overline a}\right) \to \pi \left(\Godel{\overline{\pi (\Godel{\overline a})}}\right)$
\item $\vdash \pi \left(\Godel{\overline a}\right) \to
    \pi \left(\Godel{\overline{(a \to b)}}\right) \to \pi \left(\Godel{\overline{b}}\right)$
\end{enumerate}
\subsection{Лемма о самоприменении}
\label{sec-2-37}
$a(x)$ -- формула, тогда $\exists b$ такой что
\begin{enumerate}
    \item $\vdash a\left(\Godel{\overline b}\right) \to b$
    \item $\vdash \beta \to a\left(\Godel{\overline b}\right)$
\end{enumerate}
\subsection{Вторая теорема Гёделя о неполноте ФА}
\label{sec-2-38}
Если теория непротиворечива, в ней $\nvdash Consis$
\subsection{Теория множеств}
\label{sec-2-39}
Теория множеств -- теория первого порядка, в которой
есть единственный предикат $\in$ (в ФА был =), есть связка
$\leftrightarrow$, есть пустое множество, операции пересечения и
объединения.
$x \cap y = z$, тогда $\forall t(t \in z \leftrightarrow t \in x \& t \in y)$
$x \cup y = z$, тогда $\forall t(t \in z \leftrightarrow t \in x \lor t \in y)$
$D_j(x) \forall a \forall b(a \in x \& b \in x \& a \ne  b \to a \cap b = \emptyset)$
\subsection{ZFC}
\label{sec-2-40}
\subsubsection{Аксиома равенства}
\label{sec-2-40-1}
$\forall x \forall y \forall z((x = y \& y \in z) \to x \in z)$
Eсли два множества равны, то любой элемент лежащий в первом,
лежит и во втором
\subsubsection{Аксиома пары}
\label{sec-2-40-2}
$\forall x \forall y (\lnot (x=y) \to \exists p(x \in p \& y \in p \& \forall z(z \in p \to (x = z \lor y = z))))$
$x \ne  y$, тогда сущ. $\lbrace x, y \rbrace$
\subsubsection{Аксиома объединений}
\label{sec-2-40-3}
$\forall x(\exists y(y\in x) \to \exists p \forall y(y \in p \leftrightarrow \exists s(y \in s \& s \in x)))$
Eсли x не пусто, то из любого семейства множеств можно
образовать <<кучу-малу>>, то есть такое множество p,
каждый элемент y которого принадлежит по меньшей мере
одному множеству s данного семейства s x
\subsubsection{Аксиома степени}
\label{sec-2-40-4}
$\forall x \exists p \forall y(y \in p \leftrightarrow y \in x)$
P(x) -- множество степени x (не путать с 2ˣ -- булеаном)
Это типа мы взяли наш x, и из его элементов объединением и
пересечением например понаобразовывали кучу множеств, а потом
положили их в p.
\subsubsection{Схема аксиом выделения}
\label{sec-2-40-5}
$\forall x \exists b\forall y(y \in b \leftrightarrow (y \in x \& \phi(y)))$
Для нашего множества x мы можем подобрать множество побольше,
на котором для всех элементов, являющихся подмножеством x
выполняется предикат.
\subsubsection{Аксиома выбора (не входит в ZF по дефолту)}
\label{sec-2-40-6}
Если $a = Dj(x)$ и $a \ne  0$, то $x \in a \ne  0$
\subsubsection{Аксиома бесконечности}
\label{sec-2-40-7}
$\exists N(\emptyset \in N \& \forall x(x \in N \to x \cup \{x\} \in N))$
\subsubsection{Аксиома фундирования}
\label{sec-2-40-8}
$\forall x(x = \emptyset \lor \exists y(y \in x \& y \cap x = \emptyset))$
$\forall x(x \ne  \emptyset \to \exists y(y \in x \& y \cap x = \emptyset))$
Равноценные формулы.

Я бы сказал, что это звучит как-то типа
<<не существует бесконечно вложенных множеств>>
\subsubsection{Схема аксиом подстановки}
\label{sec-2-40-9}
$\forall x \exists !y.\phi(x,y) \to \forall a\exists b\forall c(c \in b \leftrightarrow (\exists d.(d \in a \& \phi(d, c))))$
Пусть формула $\phi$ такова, что для при любом $x$ найдется единственный $y$
такой, чтобы она была истинна на $x$, $y$, тогда для любого $a$
найдется множество $b$, каждому элементу которого $c$ можно сопоставить
подмножество $a$ и наша функция будет верна на нем и на $c$
Типа для хороших функций мы можем найти множество с отображением из
его элементов в подмножество нашего по предикату.

\subsection{Ординальные числа, операции}
\label{sec-2-41}
\begin{itemize}
\item Определение вполне упорядоченного множества (фундированное
с линейныи порядком).
\item Определение транзитивного множества
Множество X транзитивно, если
$\forall a \forall b((a \in b \& b \in x) \to a \in x)$
\item Ординал -- транзитивное вполне упорядоченное отношением $\in$ мн-во
\item Верхняя грань множества ординалов S
$C | \{C = min(X) \& C \in X \mid X = \{z \mid \forall (y\in S)(z \ge y)\}\}$
$C = Upb(S)$
$Upb(\{\emptyset\}) = \{\emptyset\}$
\item Successor ordinal (сакцессорный ординал?)
Это $b = a' = a \cup \{a\}$
\item Предельны ординал
Ординал, не являющийся ни 0 ни successor'ом.
\item Недостижимый ординал
$\epsilon$ -- такой ординал, что $\epsilon  = w^\epsilon $

$\epsilon_0$ = $\operatorname{Upb}(w, w^w, w^{w^w}, w^{w^{w^w}}, \dotsc)$ -- минимальный из $\epsilon$
\item Канторова форма -- форма вида ∑(a*w$^{\text{b}}$+c), где b -- ординал, последовательность
строго убывает по b. Есть слабая канторова форма, где вместо $a (a \in N)$
пишут $a$ раз $w^b$. В канторовой форме приятно заниматься сложениями и
прочим, потому что всякие upb -- слишком ниочем.

\begin{align*}
x + 0      &= x \\
x + c'     &= (x + c)' \\
x + \lim(a) &= \operatorname{Upb}\{x + c \mid c < a\} \\
x * 0      &= 0 \\
x * c'     &= x * c + x \\
x * \lim(a) &= \operatorname{Upb}\{x * c \mid c < a\} \\
x ^ 0      &= 1 \\
x ^ {c'}     &= (x ^ c) * x \\
x ^ {\lim(a)} &= \operatorname{Upb}\{x ^ c \mid c < a\}
\end{align*}
\end{itemize}
\subsection{Кардинальные числа, операции}
\label{sec-2-42}
\begin{definition}
    Будем называть множества равномощными, если найдется биекция.
\end{definition}
\begin{definition}
    Будем называть A не превышающим по мощности B, если найдется
    инъекция $A \to B (|A| \le |B|)$
\end{definition}
\begin{definition}
    Будем называть $А$ меньше по мощности, чем $B$, если $|A| \le |B| \& |A| \ne  |B|$
\end{definition}
\begin{definition}
    Кардинальное число -- число, оценивающее мощность множества.
\end{definition}
\begin{definition}
    Кардинальное число $\aleph$ -- это ординальное число a, такое что
    $\forall x \leq a \vert x \vert \leq \vert a \vert$

    $\aleph_0 = w$ по определению; $\aleph_1 = {}$ минимальный кардинал, следующий за $\aleph_0$
\end{definition}
\begin{definition}
    Кардинальное число $\beth$ -- это ординальное число а, такое что
    $\beth_i = P(\beth_{i-1})$\\
    $\beth_0 = \aleph_0$\\
    $+: |A| + |B| = \max(|A|, |B|) \text{(если нет общих элементов)} = |A \cup B|$
\end{definition}
\subsection{Диагональный метод, теорема Лёвенгейма-Скулема}
\label{sec-2-43}
Диагональный метод -- метод доказательства $\vert$2$^{\text{X}}\vert > \vert X \vert$
\subsection{Парадокс Скулема}
\label{sec-2-44}
Мнимый парадокс, базирующийся на теореме Лёвенгейма-Скулема
и том факте, что в формальной арифметике существуют несчетные
множества. Заковырка в том, что <<существует счетное мн-во>> выражается
в ФА <<не существует биекции>>. И тогда прийти к противоречию
нельзя.
\subsection{Теорема Генцена о непротиворечивости ФА}
\label{sec-2-45}
Ну типа мы можем обернуть ФА в теорию покруче, доказать что в ней
невозможно доказать $0=1$, а потом доказать, что если $S_\infty$ непротиворечива,
то и $S$ непротиворечива.
