\section{Ticket 6: Полнота исчисления предикатов}
\label{sec-8}
Тут можно почитать конспект Д.Г.
\subsection{Свойства противоречивости}
\label{sec-8-1}
Противоречивая теория – теория, в которой можно вывести p, ¬p.
\begin{lemma}
Теория противоречива $\Leftrightarrow$ в ней выводится $a \& \neg a$
\end{lemma}
\begin{proof}
$\Leftarrow$ Если выводится $a \& \neg a$, то противоречива -- очевидно через аксиомы\\
$\Rightarrow$ Если противоречива, то выводится $a \& \neg a$\\
\begin{tabular}{lll}
(1)& $\neg \alpha$& Допущение\\
(2)& $\alpha$& Допущение\\
(3)& $\alpha \to \neg \alpha \to (\alpha \& \neg \alpha)$& Сх. акс. 10\\
(4)& $\neg \alpha \to (\alpha \& \neg \alpha)$& M.P. 1,3\\
(5)& $\alpha \& \neg \alpha$& M.P. 2,4\\
\end{tabular}\\
\end{proof}
Заметим, что всякое подмножество непротиворечивого множества непротиворечиво.\\
Заметим, что всякое бесконечное прот. множество содержит конечное противоречивое подмножество ввиду конечности вывода.\\
Совместное множество – множество с моделью (все формулы множества верны в какой-либо интерпретации).
\subsection{Лемма о дополнении непротиворечивого множества}
\label{sec-8-2}
\begin{lemma}
Для всякого непротиворечивого множества $\Gamma$ замкнутых формул сигнатуры $\sigma$ существует множество $\Gamma’$, являющееся к тому же полным, имеющее ту же сигнатуру и содержащее $\Gamma$.
\end{lemma}
\begin{proof}
Для не более чем счетных сигнатур:\\
Давайте добавлять недостающие формулы в $\Gamma$ -- если есть формула $\alpha$, добавим $\alpha$ или $\neg \alpha$ в зависимости от того, является ли $\Gamma \cup \alpha$ или $\Gamma \cup \neg \alpha$ противоречивым или нет (выберем непротиворечивый вариант). Одно всегда верно, потому что:
\begin{enumerate}
\item $\Gamma \cup \alpha$, $\Gamma \cup \neg \alpha$ противоречивы обе $\Rightarrow$ Мы можем доказать, что $\Gamma$ изначально было противоречиво
\item $\Gamma \cup \alpha$, $\Gamma \cup \neg \alpha$ не противоречивы обе $\Rightarrow$ Тогда можно сказать, что $\alpha \to \neg \alpha \to \alpha \& \neg \alpha$.
\end{enumerate}
\end{proof}
\subsection{Условие о интерпретации непротиворечивого мн-ва}
\label{sec-8-3}
Будем называть интерпретацией непротиворечивого множества формул функцию оценки, тождественно равную $1$ на элементах из этого множества. Будем говорить, что $\Gamma \vDash \alpha$, если она тождественна в любой модели $\Gamma$.
\subsection{Несколько лемм}
\label{sec-8-4}
\begin{lemma}
$\Gamma \vdash \alpha \Rightarrow \Gamma \vDash \alpha$
\end{lemma}
\begin{proof}
Механическая проверка аксиом
\end{proof}
\begin{lemma}
Eсли у $\Gamma$ есть модель, то $\Gamma$ непротиворечиво
\end{lemma}
\begin{proof}
Пусть $\Gamma$ имеет модель, но противоречиво, тогда из $\Gamma$ выводится $\alpha, \neg \alpha$, по корректности $\Gamma \vDash \alpha, \neg \alpha$, но формула и ее отрицание не могут быть общезначимыми одновременно.
\end{proof}
\begin{lemma}
Пусть $\Gamma$ -- полное непротиворечивое множество бескванторных формул. Тогда существует модель для $\Gamma$.
\end{lemma}
\begin{proof}
Построим модель структурной индукцией по формулам.\\
Предметное множество -- строки, содержащие выражения.\\
Например $\llbracket c_1 \rrbracket = \text{<<$c_1$>>}$,
$\llbracket f_1 (c_1, f_2(c_2)) \rrbracket = \text{<<$f_1 (c_1, f_2(c_2))$>>}$

Мы не хотим заниматься подсчетом, а предпочитаем оставлять то, что нужно вычислить как отдельную функцию. Рассмотрим формулу -- предикат. Его оценка истина, если он принадлежит носителю, ложна если его отрицание в носителе (в предметном множестве). Элементы всегда входят противоречиво (элемент не вдохит со своим отрицанием. Связки определим естественным образом. Докажем, что $\gamma \in \Gamma \Leftrightarrow \gamma$ истинна ($\Gamma$ -- предметное множество)
\begin{itemize}
\item База:\\
Если атомарная формула лежит в $\Gamma$, то она истинна по определению.\\
Если атомарная формула истинна, то лежит в $\Gamma$
\item Переход:
\begin{enumerate}
\item $\alpha \& \beta$

Если $\alpha \& \beta$ лежит в $\Gamma$, то оно истинно по определению
\begin{itemize}
\item Пусть $\llbracket \alpha \& \beta \rrbracket = \texttt{И}$, тогда покажем, что $\alpha \& \beta \in \Gamma$.\\
По таблице истинности $\&$ ясно, что $\llbracket \alpha \rrbracket = \texttt{И}$ и $\llbracket \beta \rrbracket = \texttt{И}$. Тогда $\alpha$ и $\beta$ лежат в $\Gamma$ по индукционному предположению. Тогда с помощью $\alpha \to \beta \to \alpha \& \beta$ можно показать, что и $\alpha \& \beta \in \Gamma$.
\item Пусть $\llbracket \alpha \& \beta \rrbracket = \texttt{Л}$, тогда покажем, что $\neg(\alpha \& \beta) \in \Gamma$.\\
По таблице истинности $\&$ ясно, что $\llbracket \alpha \rrbracket = \texttt{Л}$ или $\llbracket \beta \rrbracket = \texttt{Л}$. Для определенности возьмем, что $\alpha$ -- ложь. Тогда $\neg \alpha$ лежат в $\Gamma$ по индукционному предположению.\\
Докажем, что $\neg \alpha \vdash \neg (\alpha \& \beta)$:\\
\begin{tabular}{lll}
(1) & $\neg\alpha$ & Предположение\\
(2) & $\neg\alpha \rightarrow \alpha\&\beta\rightarrow\neg\alpha$ & Сх. акс. 1\\
(3) & $\alpha\&\beta \rightarrow \neg\alpha$ & M.P. 1,2\\
(4) & $\alpha \& \beta \rightarrow \alpha$ & Сх. акс. 4\\
(5) & $(\alpha \& \beta \rightarrow \alpha) \rightarrow (\alpha \& \beta \rightarrow \neg\alpha) \rightarrow \neg(\alpha \& \beta)$ & Сх. акс. 9\\
(6) & $(\alpha \& \beta \rightarrow \neg\alpha) \rightarrow \neg(\alpha \& \beta)$ & M.P. 5,4\\
(7) & $\neg(\alpha \& \beta)$ & M.P. 6,3
\end{tabular}
\end{itemize}
\item $\alpha \vee \beta$
\begin{itemize}
\item $\llbracket \alpha \vee \beta \rrbracket = \texttt{И}$. Тогда по таблице истинности $\vee$ либо $\llbracket \alpha \rrbracket = \texttt{И}$, либо $\llbracket \beta \rrbracket = \texttt{И}$. Не умаляя общности скажем, что $\llbracket \alpha \rrbracket = \texttt{И}$. Тогда $\alpha \in \Gamma$ по предположению индукции. Легко можно доказать, что и $\alpha \vee \beta \in \Gamma$ с помощью $\alpha \to \alpha \vee \beta$.
\item $\llbracket \alpha \vee \beta \rrbracket = \texttt{Л}$. Тогда по таблице истинности $\vee$ и $\llbracket \alpha \rrbracket = \texttt{Л}$, и $\llbracket \beta \rrbracket = \texttt{Л}$. Тогда $\neg \alpha \in \Gamma$ и $\neg \beta \in \Gamma$ по предположению индукции. С помощью 9-ой схемы аксиом мы можем доказать, что и $\neg (\alpha \vee \beta) \in \Gamma$.
\end{itemize}
\item Аналогично нужно доказать все связки
\end{enumerate}
\end{itemize}
\end{proof}
\subsection{Построение \texorpdfstring{$\Gamma ^ *$}{Г*}}
\label{sec-8-5}
\begin{theorem}
Можно построить из нашего множества формул множество бескванторных формул
\end{theorem}
\begin{proof}
Для этого определим такую операцию избавления от 1 квантора: Построим новый язык, отличающийся от нашего контантами, там будут $d_i^j$, где нижний индекс -- это поколение, верхний – нумерационный. Возьмем непротиворечивое множество формул $\Gamma_i$ и пополним его, получив непротиворечивое множество формул $\Gamma_{i+1}$, такое что $\Gamma_i \subset \Gamma_{i+1}$. Возьмем формулу $\gamma \in \Gamma_{i}$. Рассмотрим случаи:
\begin{enumerate}
\item Не содержит кванторов\\
Тогда делать ничего не нужно
\item $\gamma = \forall x (a)$\\
Тогда возьмем все константы, использующиеся в $\Gamma_{i}$ -- это будут $c_i$, $d_a^j$, где $a \leq i$. Занумеруем их $\theta_1, \theta_2, \dots$. И добавим формулы $a_1=a[x:=\theta_1], \dots$ к $\Gamma_{i+1}$.
\item $\gamma = \exists x (a)$\\
Тогда возьмем новую константу $d_{i+1}^j$ и добавим $a[x:=d_{i+1}^j]$ к $\Gamma_{i+1}$.
\end{enumerate}

Заметим, что сами формулы с кванторами мы не выкидываем -- ведь в будущем появятся новые формулы, и процесс для уже использованных кванторных формул нужно будет повторить. Покажем, что полученные множества остаются непротиворечивыми. $\Gamma_i$ непротиворечиво, а $\Gamma_{i+1}$ противоречиво, тогда $\Gamma_{i+1} \vdash \alpha \& \neg \alpha$, тогда выпишем конечное доказательство, найдем посылки, новые в $\Gamma_{i+1}$, которых нету в $\Gamma_{i}$, выпишем их и впихнем направо по теореме о дедукции: $\Gamma_{i} \vdash \gamma_1 \to \gamma_2 \to \gamma_3 \to \dots \to \gamma_n \to \beta \& \neg \beta$ Новые посылки у нас получаются только из пунктов 2 и 3.

\begin{enumerate}
\item $\gamma_1 = a[x:=\theta_1]$ из $\forall x (a)$. Тогда рассмотрим доказательство:\\
\begin{tabular}{lll}
$(1)$ & $\forall x \alpha \rightarrow \alpha [x := \theta]$ & Сх. акс. $\forall$\\
$(2)$ & $\forall x \alpha$ & $\forall x \alpha$ из $\Gamma_g$\\
$(3)$ & $\alpha [x := \theta]$ & M.P. $2,1$\\
$(4 \dots k)$ & $\alpha [x := \theta] \rightarrow (\gamma_2 \rightarrow \dots \gamma_n \rightarrow \beta \& \neg \beta)$ & Исх. формула\\
$(k+1)$ & $\gamma_2 \rightarrow \dots \gamma_n \rightarrow \beta \& \neg \beta$ & M.P. $3,k$
\end{tabular}
\item $\gamma_1 = a[x:=d_{i+1}^k]$ из $\exists x(a)$ выберем переменную, не участвующую в выводе противоречия -- $z$.
    Заменим все вхождения $d^k$ в д-ве на $z$.
    Поскольку $d_{i+1}^k$ -- константа, мы можем делать такие замены.
    Поскольку $z$ -- константа, специально введенная для замены и раньше не встречавшаяся, то она
    отсутствует в $\gamma_2,\dots$ + мы можем правильно выбрать $b$, чтобы и в нем отсутствовала $d_{i+1}^k$.
    Значит мы можем применить правило для выведения $\exists$:\\
\begin{tabular}{lll}
$(1 \dots k)$ & $\alpha [x := y] \rightarrow (\gamma_2 \rightarrow \dots \gamma_n \rightarrow \beta \& \neg \beta)$ & Исх. формула\\
$(k+1)$ & $\exists y \alpha [x := y] \rightarrow (\gamma_2 \rightarrow \dots \gamma_n \rightarrow \beta \& \neg \beta)$ & Правило для $\exists$\\
$(k+2)$ & $\exists x \alpha$ & Т.к. $\exists x \alpha$ из $\Gamma_g$ \\
$(k+3 \dots l)$ & $\exists y \alpha [x := y]$ & Доказуемо \\
$(l+1)$ & $\gamma_2 \rightarrow \dots \gamma_n \rightarrow \beta \& \neg \beta$ & M.P. $l, k+1$
\end{tabular}
\end{enumerate}
Возьмем $\Gamma_0 = \Gamma$. $\Gamma^* = \cup \Gamma_i$. $\Gamma^*$ также не противоречиво, потому что д-во использует конечное количество предположений, добавленных на каком-то шаге $j$ максимум, значит множество $Γ_j$ тоже противоречиво, что невозможно по условию.
\end{proof}
\subsection{Доказательство того, что дополненное бескванторное подмножество \texorpdfstring{$\Gamma^*$}{Г*} -- модель для \texorpdfstring{$\Gamma$}{Г}}
\label{sec-8-6}
\begin{theorem}
Дополненное бескванторное подмножество $\Gamma^*$ -- модель для $\Gamma$
\end{theorem}
\begin{proof}
Выделим в $\Gamma^*$ бескванторное подмножество $G$. Пополним его по лемме 2 (лемма о дополнении непротиворечевиого множества) модель сделаем из него по лемме о бескванторной модели. Покажем, что это модель для всего $\Gamma^*$, а значит и для $\Gamma$. Рассмотрим $\gamma \in \Gamma^*$, покажем, что $[\gamma] = \texttt{И}$.
\begin{itemize}
\item База\\
Формула не содержит кванторов. Истинность гарантируется леммой о бескванторном множестве.
\item Переход\\
Пусть $G$ это модель для любой формулы из $\Gamma^*$ с $r$ кванторами, покажем что она остается моделью для $r+1$ квантора.
\begin{enumerate}
\item $\gamma = \forall x (a)$\\
Покажем, что формула истинна для любого $t \in D$. По построению подели есть такое $\theta$, что $t = "\theta$ (string). По построению $\Gamma^*$ начиная с шага $p+1$ мы добавляем формулы вида $a[x:=k]$, где $k$ -- конструкция из констант и ф.симв. Также каждая константа ($c_i$ или $d_i^j$) из $\theta$ добавлена на некотором шаге $s_k$. То есть будет шаг $l=max(max(s_k), p)$, на котором $\theta$ обретет смысл и в $\Gamma_{l+1}$ будет присутствовать $a[x:=\theta]$. В формуле $a$ на один квантор меньше, значит она истинна по предположению индукции.
\item $\gamma = \exists x (a)$\\
По построению $\Gamma^*$ как только добавили $a$ к $\Gamma_i$, так сразу в следующем мире $\Gamma_{i+1}$ появляется $a[x:=d_{i+1}^k]$. Значит формула истинна на значении $"d_{i+1}^k"$, то есть истинна.
\end{enumerate}
\end{itemize}
\end{proof}
\subsection{Следствие -- если \texorpdfstring{$\vDash \alpha$}{⊨ a}, то \texorpdfstring{$\vdash \alpha$}{⊢ a}}
\label{sec-8-7}
\begin{theorem}
$\vDash \alpha \Rightarrow \vdash \alpha$
\end{theorem}
\begin{proof}
\begin{itemize}
\item Пусть $\Gamma \nvdash a$, тогда по полноте множества $\Gamma$, $\Gamma \vdash \neg a$, но у $\Gamma$ есть модель, в которой $\Gamma \vDash \neg a$. То есть $\Gamma \nvDash a$. Но $\Gamma$ по построению то же, что и модель теории, то есть все рассуждения $\Gamma \vdash a$ равноценны в предикатах $\vdash a$.
\item Пусть $\nvdash a$, тогда пусть $\Gamma=\lbrace \neg a \rbrace$
\begin{enumerate}
\item $\Gamma$ непротиворечиво\\
Пусть $\Gamma$ противоречиво, значит $\forall b \Gamma \vdash b, \Gamma \vdash \neg b$;
\begin{enumerate}[label=(\alph*)]
\item $\neg a \vdash b, \neg a \vdash b$;
\item $\neg a \vdash a, \neg a \vdash \neg a$;
\item $\vdash \neg a \to a, \neg a \to \neg a$;
\item $\vdash (\neg a \to a) \to (\neg a \to \neg a) \to \neg \neg a$;
\item $\vdash \neg \neg a \to a$;
\item $\vdash a \to\leftarrow а$ недоказуемо по условию.;
\end{enumerate}
\item $\Gamma$ подходит под условие теоремы Гёделя о полноти исчисления предикатов, то есть у $\Gamma$ есть модель. Тогда в ней оценка $[\neg a] = 1$, значит оценка $[a] = 0$, то есть $\nvDash a$. Мы доказали мета-контрпозицию $\nvdash a \Rightarrow \nvDash a$.
\end{enumerate}
\end{itemize}
\end{proof}
