\section{Ticket 7: ФА}
\label{sec-9}
\subsection{Структуры и модели, теория первого порядка}
\label{sec-9-1}
Теория первого порядка - это формальная система с кванторами по
функциональным символам, но не по предикатам. Рукомахательное
определение – это фс с логикой первого порядка в основе, в которой
абстрактные предикаты и функциональные символы определяются точно
(а может такое определение даже лучше).

Структура по ДГ:

Структурой теории первого порядка мы назовем упорядоченную тройку
$\langle D, F, P\rangle$, где F -- списки оценок для 0-местных, 1-местных и т.д.
функций, и $P = P_0, P_1,\dotsc$ -- списки оценок для 0-местных,
1-местных и т.д. предикатов, $D$ — предметное множество.

Понятие структуры — развитие понятия оценки из исчисления предикатов.
Но оно касается только нелогических составляющих теории; истинностные
значения и оценки для связок по-прежнему определяются исчислением
предикатов, лежащим в основе теории. Для получения оценки формулы
нам нужно задать структуру, значения всех свободных индивидных
переменных, и (естественным образом) вычислить результат.

Структура по-моему:

Все то же самое определение из ИВ. Мы просто забиваем на предикаты
в ИВ (не определям их), расширяем нашу сигнатуру (добавляя конкретные
предикаты и функциональные символы), определяем для нее интерпретацию.

Модель -- это корректная структура (любое доказуемое утверждение должно
быть в ней общезначимо).
\subsection{Аксиомы Пеано}
\label{sec-9-2}
Множество $N$ удовлетворяет аксиоматике Пеано, если:
\begin{enumerate}
\item $0 \in N$
\item $x \in N, succ(x) \in N$
\item $\nexists x \in N : (S(x) = 0)$
\item $(succ(a) = c \& succ(b) = c) \to a = b$
\item $P(0) \& \forall n.(P(n) \to P(succ(n))) \to \forall n.P(n)$
\end{enumerate}
\subsection{Формальная арифметика -- аксиомы, схемы, правила вывода}
\label{sec-9-3}
Формальная арифметика -- это теория первого порядка, у которой
сигнатура определена как: (циферки, логические связки, алгебр.
связки, '), а интерпретацию сейчас будем определять.
Интерпретация определяет два множества -- $V, P$ -- истинностные и
предметные значения. На самом деле нет никакого множества P,
мы определяем только $V$, потому что оно нужно для оценок. Все
элементы, которые мы хотели бы видеть, выражаются в сигнатуре.
Пусть множество $V = \lbrace 0, 1 \rbrace$ по-прежнему.
Определим оценки логических связок естественным образом.
Определим алгебраические связки так:
\begin{align*}
    +(a, 0 ) &= a \\
    +(a, b') &= (a + b)' \\
    *(a, 0 ) &= 0 \\
    *(a, b') &= a * b + a
\end{align*}

\textbf{Тут должно быть что-то на уровне док-ва $2+2=4$}
\subsubsection{Аксиомы}
\label{sec-9-3-1}
\begin{enumerate}
\item $a = b \to a' = b'$
\item $a = b \to a = c \to b = c$
\item $a' = b' \to a = b$
\item $\lnot (a' = 0)$
\item $a + b' = (a + b)'$
\item $a + 0 = a$
\item $a * 0 = 0$
\item $a * b' = a * b + a$
\item $\phi[x:=0] \& \forall x.(\phi \to \phi[x:=x']) \to \phi$
\end{enumerate}
\subsubsection{a = a}
\label{sec-9-3-1-1}
\begin{lemma}
$\vdash a = a$
\end{lemma}
\begin{proof}
$\vdash a = a$\\
% Работает — не трожь ©
\begin{tabular}{@{}lll}
(1)& $a = b \to a = c \to b = c$& Сх. акс. ФА 2\\
(2)& $T$& Сх. акс.\\
(3)& $(a = b \to a = c \to b = c) \to T \to (a = b \to a = c \to b = c)$& Сх. акс. 1\\
(4)& $T \to (a = b \to a = c \to b = c)$& M.P. 1,3\\
(5)& $T \to \forall a (a = b \to a = c \to b = c)$& ПВ $\forall$\\
(6)& $T \to \forall a \forall b (a = b \to a = c \to b = c)$& ПВ $\forall$\\
(7)& $T \to \forall a \forall b \forall c (a = b \to a = c \to b = c)$& ПВ $\forall$\\
(8)& $\forall a \forall b \forall c (a = b \to a = c \to b = c)$& M.P. 2,7\\
(9)& \begin{tabular}[t]{@{}l}$\forall a \forall b \forall c (a = b \to a = c \to b = c) \to$\\
\hspace{5cm}$\forall b \forall c (a + 0 = b \to a + 0 = c \to b = c)$\end{tabular}& \begin{tabular}[t]{@{}l}\\Сх. акс. ИП 1\end{tabular}\\
(10)& $\forall b \forall c (a + 0 = b \to a + 0 = c \to b = c)$& M.P. 8,9\\
(11)& \begin{tabular}[t]{@{}l}$\forall b \forall c (a + 0 = b \to a + 0 = c \to b = c) \to$\\
\hspace{5cm}$(\forall c (a + 0 = a \to a + 0 = c \to a = c))$\end{tabular}& \begin{tabular}[t]{@{}l}\\Сх. акс. ИП 1\end{tabular}\\
(12)& $\forall c (a + 0 = a \to a + 0 = c \to a = c)$& M.P. 10,11\\
(13)& \begin{tabular}[t]{@{}l}$(\forall c (a + 0 = a \to a + 0 = c \to a = c)) \to$ \\
\hspace{5cm}$(a + 0 = a \to a + 0 = a \to a = a)$\end{tabular}& \begin{tabular}[t]{@{}l}\\Сх. акс. ИП 1\end{tabular}\\
(14)& $a + 0 = a \to a + 0 = a \to a = a$& M.P. 12,13\\
(15)& $a + 0 = a$& Сх. акс. ФА 6\\
(16)& $a + 0 = a \to a = a$& M.P. 15,14\\
(17)& $a = a$& M.P. 15,16\\
\end{tabular}\\
\end{proof}
