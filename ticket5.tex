\section{Ticket 5: Логика 2 порядка}
\label{sec-7}
\subsection{Основные определения}
\label{sec-7-1}
Смотрим коснпект ДГ
\subsection{Теорема о дедукции}
\label{sec-7-2}
\begin{theorem}
Если $\Gamma, \alpha \vdash \beta$, и в доказательстве отсутствуют применения правил для кванторов, использующих свободные переменные  из формулы $\alpha$, то $\Gamma \vdash \alpha \rightarrow \beta$
\end{theorem}
\begin{proof}
Будем рассматривать формулы в порядке сверху вниз. На $i$-ой строке встретили формулу $\delta_{i}$. Тогда докажем, что $\alpha \rightarrow \delta_{i}$. Разберем случаи:
\begin{enumerate}
\item $\delta_{i}$ - старая аксиома, совпадает с $\alpha$ или выводится по правилу M.P.\\
Тогда мы знаем, что делать из Теоремы о дедукции для ИВ
\item $\delta_{i}$ - новая аксиома\\
Тогда все то же самое, что и в старой аксиоме, но нужно так же проверить условие.
\item $\exists x (\psi) \rightarrow \phi$ - новое правило вывода
\begin{itemize}
\item Докажем вспомогательную лемму:\\
\begin{lemma}
$(\alpha \rightarrow (\beta \rightarrow \gamma)) \rightarrow (\beta \rightarrow (\alpha \rightarrow \gamma))$
\end{lemma}
\begin{proof}
Докажем, что $\alpha \rightarrow (\beta \rightarrow \gamma), \beta, \alpha \vdash \gamma$:\\
\begin{tabular}{lll}
(1) & $\alpha \rightarrow \beta \rightarrow \gamma$& Допущение\\
(2) & $\alpha$& Допущение\\
(3) & $\beta \rightarrow \gamma$& M.P. 2,1\\
(4) & $\beta$& Допущение\\
(5) & $\gamma$& M.P. 4,3\\
\end{tabular}
\end{proof}
\item По индукционному преположению мы знаем, что $\alpha \rightarrow \psi \rightarrow \phi$. Тогда докажем, что $\alpha \rightarrow \psi \rightarrow \phi, (\alpha \rightarrow \psi \rightarrow \phi) \rightarrow (\psi \rightarrow \alpha \rightarrow \phi) \vdash \alpha \rightarrow \exists x (\psi) \rightarrow \phi$:\\
\begin{tabular}{lll}
(1) & $(\alpha \rightarrow \psi \rightarrow \phi) \rightarrow (\psi \rightarrow \alpha \rightarrow \phi)$& Допущение\\
(2) & $\alpha \rightarrow \psi \rightarrow \phi$& Допущение\\
(3) & $\psi \rightarrow \alpha \rightarrow \phi$& M.P. 2,1\\
(4) & $\exists x (\psi) \rightarrow \alpha \rightarrow \phi$& Правило вывода 1\\
(5) & $(\exists x (\psi) \rightarrow \alpha \rightarrow \phi) \rightarrow (\alpha \rightarrow \exists x (\psi) \rightarrow \phi)$& Допущение\\
(6) & $\alpha \rightarrow \exists x (\psi) \rightarrow \phi$& M.P. 4,5\\
\end{tabular}
\end{itemize}
\item $\phi \rightarrow \forall x (\psi)$ - новое правило вывода
\begin{itemize}
\item Докажем вспомогательную лемму 1
\begin{lemma}
$(\alpha \& \beta \rightarrow \gamma) \rightarrow (\alpha \rightarrow \beta \rightarrow \gamma)$
\end{lemma}
\begin{proof}
Докажем, что $(\alpha \& \beta \rightarrow \gamma), \alpha, \beta \vdash \gamma$:\\
\begin{tabular}{lll}
(1) & $\alpha$& Допущение\\
(2) & $\beta$& Допущение\\
(3) & $\alpha \rightarrow \beta \rightarrow \alpha \& \beta$& Сх. акс. 1\\
(4) & $\beta \rightarrow \alpha \& \beta$ M.P. 1,3\\
(5) & $\alpha \& \beta$& M.P. 2,4\\
(6) & $\alpha \& \beta \rightarrow \gamma$& Допущение\\
(7) & $\gamma$& M.P. 5,6\\
\end{tabular}
\end{proof}
\item Докажем вспомогателньую лемму 2
\begin{lemma}
$(\alpha \rightarrow \beta \rightarrow \gamma) \rightarrow (\alpha \& \beta \rightarrow \gamma)$
\end{lemma}
\begin{proof}
Докажем, что $\alpha \rightarrow \beta \rightarrow \gamma, \alpha \& \beta \vdash \gamma$:\\
\begin{tabular}{lll}
(1) & $\alpha \& \beta \rightarrow \alpha$& Сх. акс. 4\\
(2) & $\alpha \& \beta$& Допущение\\
(3) & $\alpha$& M.P. 2,1\\
(4) & $\alpha \& \beta \rightarrow \beta$& Сх. акс. 5\\
(5) & $\beta$& M.P. 2,4\\
(6) & $\alpha \rightarrow \beta \rightarrow \gamma$& Допущение\\
(7) & $\beta \rightarrow \gamma$& M.P. 3,6\\
(8) & $\gamma$& M.P. 5,7\\
\end{tabular}
\end{proof}
\item По индукционному предположению мы знаем, что $\alpha \rightarrow \psi \rightarrow \phi$. Тогда докажем, что $\alpha \rightarrow \psi \rightarrow \phi \vdash \alpha \rightarrow \psi \rightarrow \forall (\phi)$.\\
\begin{tabular}{lll}
(1) & $(\alpha \rightarrow \psi \rightarrow \phi) \rightarrow (\alpha \& \psi \rightarrow \phi)$& Вспомогательная лемма 1\\
(2) & $\alpha \rightarrow \psi \rightarrow \phi$& Допущение\\
(3) & $\alpha \& \psi \rightarrow \phi$& M.P. 2,1\\
(4) & $\alpha \& \psi \rightarrow \forall (\phi)$& Правило вывода 2\\
(5) & $(\alpha \& \psi \rightarrow \forall (\phi)) \rightarrow (\alpha \rightarrow \psi \rightarrow \forall (\phi))$& Вспомогательная лемма 2\\
(6) & $\alpha \rightarrow \psi \rightarrow \forall (\phi)$& M.P. 4,5\\
\end{tabular}
\end{itemize}
\end{enumerate}
\end{proof}
\subsection{Корректность исчисления предикатов}
\label{sec-7-3}
Смотрим конспект ДГ