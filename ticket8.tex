\section{Ticket 8: рекурс, Аккерман}
\label{sec-10}
\subsection{Рекурсивные функции}
\label{sec-10-1}
$Z(x) = 0$\\
$N(x) = x + 1$\\
$U^n_i(x_1\ldots{}x_n) = x_i$\\
$S<f, g_1\ldots{}g_n>(x_1\ldots{}x_m) = f(g_1(x_1\ldots{}x_m),\ldots{}g_n(x_1\ldots{}x_m))$\\
$R<f, g>(x_1\ldots{}x_n, n) = if n = 0
f(x_1\ldots{}x_n)
else g(x_1\ldots{}x_n, n, R<f, g>(x_1\ldots{}x_n, n - 1))$\\
$μ<f>(x_1\ldots{}x_n) - минимальное k, такое что f(x_1\ldots{}x_n, k) = 0$

Пример:
$a + b = R<U^2_1, S<N, U^3_3>>(a, b)$
\subsection{Характеристическая функция и рекурсивное отношение}
\label{sec-10-2}
\begin{itemize}
\item \emph{Характеристическая фукнция} от выражения возвращает 1 если
выражение истинно, 0 иначе.
\item \emph{Рекурсивное отношение} - отношение, характеристическая функция
которого рекурсивна.
\end{itemize}
\subsection{Аккерман не примитивно-рекурсивен, но рекурсивен (второе)}
\label{sec-10-3}
Функция Аккермана это функция, удовлетворяющая следующим правилам:\\
\begin{itemize}
\item $A(0, n) = n + 1$
\item $A(m, 0) = A(m - 1, 1)$
\item $A(m, n) = A(m - 1, A(m, n - 1))$
\end{itemize}
Например:\\
$A(2, 0) = A(1, 1) = A(0, A(1, 0)) = A(0, 2) = 3$
\begin{lemma}
$A(m, n) \geq 1$
\end{lemma}
\begin{proof}
$A(m, n)$ определена только на натуральных числах,
$A(0, 0) = 1, A(1, 0) = A(0, 1) = 2, A(0, 1) = 2$,
все остальное еще больше
\end{proof}
\begin{lemma}
$A(1, n) = n + 2$
\end{lemma}
\begin{proof}
$A(1, n)
= A(0, A(1, n - 1))
= A(0, A(0, A(1, n - 2)))
= A(0, A(0, A(0, \ldots{} A(1, 0))))
= A(0, A(0, A(0, \ldots{} 2)))
= n + 2$ ($n$ раз инкрементируем двойку)
\end{proof}
\begin{lemma}
$A(2, n) = 2n + 3$
\end{lemma}
\begin{proof}
$A(2, n)
= A(1, A(1, \ldots{} A(2, 0)))
= A(1, A(1, \ldots{} 3))
= 2n + 3$ ($n$ раз к тройке прибавляем $A(0, 1) = 2$)
\end{proof}
\begin{lemma}
$A(m, n) \geq n + 1$
\end{lemma}
\begin{proof}
В первом случае $A \geq n + 1 = n + 1$
Во втором $A$ может перейти в первый случай, который работает
хорошо, или в третий.
В третьем случае мы можем получить $A(0, n)$ если первый аргумент
был нулем, тогда все ок, можем получить $A(1, 0)$, тогда это второй
случай, для него условие выполнено.
Третий ссылается на второй, а второй на третий, но тут
нет противоречия, потому что мы знаем, что функция Аккермана
завершается.
\end{proof}
\begin{lemma}
$A(m, n) < A(m, n + 1)$
\end{lemma}
\begin{proof}
индукция по $m$:
\begin{itemize}
\item база
$A(0, n) = n + 1 < n + 2 = A(0, n + 1)$
\item переход:
$A(k + 1, m) < A(k + 1, m) + 1$
$\geq A(k, A(k + 1, m))$ (по лемме 2)
$\geq A(k + 1, m + 1)$   (iii)
\end{itemize}
\end{proof}
\begin{lemma}
$A(m, n + 1) \leq A(m + 1, n)$
\end{lemma}
\begin{proof}
индукция по $n$:
\begin{itemize}
\item база
$A(m, 0 + 1) = A(m, 1) = A(m + 1, 0)$ (ii)
\item переход, предположение: $A(m, j + 1) \leq A(m + 1, j)$
по лемме 2 $(j + 1) + 1 \leq A(m, j + 1)$
$A(m, (j + 1) + 1) \leq A(m, A(m, j + 1))$ (по монотонности)
$A(m, A(m, j + 1)) \leq A(m, A(m + 1, j))$ (по монотонности + предположение)
$A(m, (j + 1) + 1) \leq A(m, A(m + 1, j)) = A(m + 1, j + 1)$ (iii)
\end{itemize}
\end{proof}
\begin{lemma}
$A(m, n) < A(m + 1, n)$
\end{lemma}
\begin{proof}
$A(m, n) < A(m, n + 1) \leq A(m + 1, n)$ (3а, 3b)
\end{proof}
\begin{lemma}
$A(m_1, n) + A(m₂, n) < A(max(m_1, m₂) + 4, n)$
\end{lemma}
\begin{proof}
$A(m_1, n) + A(m_2, n)
≤ A(max(m_1, m_2), n) + A(max(m_1, m_2), n)
= 2 * A(max(m_1, m_2), n)
< 2 * A(max(m_1, m_2), n) + 3
= A(2, A(max(m_1, m_2), n))$     лемма 1
$< A(2, A(max(m_1, m_2) + 3, n))$ строгая монотоннасть по обоим арг.
$< A(max(m_1, m_2) + 2, A(max(m_1, m_2) + 3, n))$ лемма 3с
$= A(max(m_1, m_2) + 3, n + 1)$   (iii)
$≤ A(max(m_1, m_2) + 4, n)$       лемма 3b
\end{proof}
\begin{lemma}
$A(m, n) + n < A(m + 4, n)$
\end{lemma}
\begin{proof}
$A(m, n) + n
< A(m, n) + n + 1
= A(n, m) + A(0, n)
< A(m + 4, n)$
\end{proof}
\begin{theorem}
Функция аккерманна не притивно-рекурсивна
\end{theorem}
\begin{proof}
TODO
\end{proof}
\iffalse
\subsubsection{Аккерманн не примитивно-рекурсивен}
\label{sec-10-3-10}
Пусть f(n_1\ldots{}nₖ) - примитивная рекурсинвная функция, k ≥ 0.
\exists J:f(n_1\ldots{}nₖ)<A(J, ∑(n_1,\ldots{}nₖ))

]n\textasciitilde{} = (n_1\ldots{}nₖ)
Индукция по рекурсивным функциям
\begin{itemize}
\item База:
f(n\textasciitilde{}) - N или Z или Uₖⱼ
\begin{enumerate}
\item f(n\textasciitilde{}) = N, k = 1; Пусть J=1, по (i) и лемме 3c
f(n) = N(n) = n + 1 = A(0, n) < A(1, n) = A(J, n) = A(J, ∑(n\textasciitilde{}))
\item f(n\textasciitilde{}) = Z, k = 1;
f(n) = 0 < A(J, n) (потому что A ≥ 1) = A(J, ∑(n\textasciitilde{}))
\item f(n\textasciitilde{}) = Uₖⱼ; k = k; Пусть J=1
f(n_1\ldots{}nₖ) = Uₖⱼ(n_1\ldots{}nₖ) = nⱼ
Пусть nⱼ = 0, тогда f(n) = 0 < A(J, ∑(n\textasciitilde{})) для любого нормального J
Пусть nⱼ > 0, тогда f(n) = (nⱼ - 1) + 1 = A(0, nⱼ - 1) < A(1, n)
= A(J, ∑(n\textasciitilde{}))
\end{enumerate}
\item Переход
\begin{enumerate}
\item Предположим, что f(n\textasciitilde{}) = S<h, g_1\ldots{}gₘ>(n\textasciitilde{}) = h(g_1(n\textasciitilde{}),\ldots{}gₘ(n\textasciitilde{}))
По предположению индукции существует J₀ для h, J_1\ldots{}Jₘ для g_1\ldots{}gₘ.
f(n\textasciitilde{}) = h(g_1(n\textasciitilde{}),..)
< A(J₀, ∑\{i=1..m\}(n\textasciitilde{}))            по выбору J₀
< A(J₀, ∑(A(Jᵢ, ∑(n\textasciitilde{}))))           по выбору Jᵢ и строгой монотонности
// J* = max(J_1..Jₘ) + 4(m - 1)
< A(J₀, A(J*, ∑(n\textasciitilde{})))             по лемме 4 примененной m-1 раз
< A(J₀, A(J*+1, ∑(n\textasciitilde{})))           по монотонности
≤ A(J₀, A(max(J₀, J*) + 1, ∑(n\textasciitilde{}))) по монотонности
≤ A(max(J₀, J*) + 1, ∑(n\textasciitilde{}) + 1)   (iii)
= A(max(J₀, J*) + 2, ∑(n\textasciitilde{}))       по лемме 3b
Тогда пусть j=max(J₀, J*) + 2
\item Пусть f(n\textasciitilde{}) = R<h,g>(n\textasciitilde{})
f(n_1\ldots{}nₖ, 0) = h(n_1\ldots{}nₖ)
f(n_1\ldots{}nₖ, m+1) = g(n_1\ldots{}nₖ, m, f(n_1\ldots{}nₖ, m))
По предположению имеем J₀ (h), J_1 (g).
] J = max(J₀, J_1) + 4
\begin{enumerate}
\item f(n\textasciitilde{}, 0)
≤ f(n\textasciitilde{}, 0) + ∑(n\textasciitilde{})
= h(n\textasciitilde{}) + ∑(n\textasciitilde{})
< A(J₀, ∑(n\textasciitilde{})) + ∑(n\textasciitilde{})
< A(J₀ + 4, ∑(n\textasciitilde{}))                   по лемме 5
< A(J, ∑(n\textasciitilde{}))                       по монотонности
= A(J, ∑(n\textasciitilde{}) + 0)
\item f(n\textasciitilde{}, k + 1)
= g(n\textasciitilde{}, k, f(n\textasciitilde{}, k))
< A(J_1, ∑(n\textasciitilde{}) + k + f(n\textasciitilde{}, k))        по выбору J_1
< A(J_1, ∑(n\textasciitilde{}) + k + 1 + f(n\textasciitilde{}, k))    по монотонности
= A(J_1, A(0, ∑(n\textasciitilde{}) + k) + f(n\textasciitilde{}, k))  (i)
< A(J_1, A(0, ∑(n\textasciitilde{}) + k) + H(J, ∑(n\textasciitilde{})+k)) по предположению
< A(J_1, A(J, ∑(n\textasciitilde{})+k)+A(J, ∑(n\textasciitilde{}) + k)) по монотонности (J > 0)
= A(J_1, 2 * [A(J, ∑(n\textasciitilde{}) + k)])
< A(J_1, 2 * [A(J, ∑(n\textasciitilde{}) + k)] + 3)
= A(J_1, A(2, A(J, ∑(n\textasciitilde{}) + k)))        по лемме 1
< A(J_1, A(J_1 + 1, A(J, ∑(n\textasciitilde{}) + k)))   по строгой монотонности (J_1 > 2)
= A(J_1 + 1, A(J, ∑(n\textasciitilde{}) + k) + 1)      (iii)
≤ A(J_1 + 2, A(J, ∑(n\textasciitilde{}) + k))
< A(J - 1, A(J, ∑(n\textasciitilde{}) + k))           по монот. J > max(..) + 4
= A(J, ∑(n\textasciitilde{}) + (k + 1))               (iii), J != 0
\end{enumerate}
\end{enumerate}
\end{itemize}
\fi
\begin{theorem}
Функция Аккерманна рекурсивна
\end{theorem}
\begin{proof}
Можем сказать, что он рекурсивный, потому что мы можем
его написать на компьютере, а тьюринг выражается в рекурсивных функциях.
\end{proof}