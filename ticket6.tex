\section{Ticket 6: Полнота исчисления предикатов}
Тут можно почитать конспект Д.Г.
\subsection{Свойства противоречивости}
Противоречивая теория – теория, в которой можно вывести p, ¬p.
\begin{lemma}
Теория противоречива $\Leftrightarrow$ в ней выводится $a \& \neg a$
\end{lemma}
\begin{proof}
$\Leftarrow$ Если выводится $a \& \neg a$, то противоречива -- очевидно через аксиомы\\
$\Rightarrow$ Если противоречива, то выводится $a \& \neg a$\\
\begin{tabular}{lll}
(1)& $\neg \alpha$& Допущение\\
(2)& $\alpha$& Допущение\\
(3)& $\alpha \to \neg \alpha \to (\alpha \& \neg \alpha)$& Сх. акс. 10\\
(4)& $\neg \alpha \to (\alpha \& \neg \alpha)$& M.P. 1,3\\
(5)& $\alpha \& \neg \alpha$& M.P. 2,4\\
\end{tabular}\\
\end{proof}
Заметим, что всякое подмножество непротиворечивого множества непротиворечиво.\\
Заметим, что всякое бесконечное прот. множество содержит конечное противоречивое подмножество ввиду конечности вывода.\\
Совместное множество – множество с моделью (все формулы множества верны в какой-либо интерпретации).